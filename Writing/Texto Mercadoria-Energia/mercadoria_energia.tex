\documentclass[12pt]{article}
\usepackage{graphicx} % Required for inserting images
\usepackage[portuguese]{babel}
\usepackage{url}
\usepackage{hyperref}

\usepackage[alf]{abntex2cite}
\bibliographystyle{abntex2-alf}

\title{Da energia enquanto mercadoria}
\author{Nicholas Funari Voltani}
\date{27 de dezembro de 2025}

\begin{document}

\maketitle

\section{}
O conceito de energia é algo que mesmo físicos do porte de Richard Feynman\footnote{Um dos ganhadores do Nobel da Física de 1965 pelas contribuições ao desenvolvimento da Eletrodinâmica Quântica (em particular pelos famosos ``diagramas de Feynman''). Uma das maiores inspirações de uma enorme parte dos estudantes de Física ao redor do mundo --- mais que Einstein, muitas vezes!} consideravam ``misteriosos'', não porque ela seja algo que a compreensão humana não consiga abarcar, e sim porque ela é não algo imediatamente palpável; não porque não se sabe o que a energia \textit{é} (pois isso é sabido muito bem), e sim por não saber-se ``\textit{de que}'' ela é feita. Vide Feynman,
\begin{quote}
    ``It is important to realize that[,] in physics today, we have no knowledge of what energy is. We do not have a picture that energy comes in little blobs of a definite amount. It is not that way. However, there are formulas for calculating some numerical quantity [i.e. total energy], and when we add it all together it gives [...] always the same number. It is an abstract thing in that it does not tell us the mechanism or the reasons for the various formulas [kinetic energy, etc.].'' (\href{https://www.feynmanlectures.caltech.edu/I_04.html}{Feynman Lectures on Physics, Vol. I, Cap. 4})
\end{quote}

O capitalista, por sorte, pode delegar a Física aos físicos e às conversas de bar, ocupando sua mente com o mais importante: ser um homem (ou mulher) eminentemente prático, ``que nem sempre sabe o que diz quando se encontra fora do seu negócio, mas [que] sabe muito bem o que faz dentro dele'' \footnote{\cite[p. 269]{Marx2017}.}. Os mistérios do universo não importam-lhe senão como capricho; o que importa-lhe, aí sim, é como tais forças naturais podem ser domadas e dobradas para dentro de seu bolso.

\section{Energia enquanto capital circulante}
Energia, no que tange aos interesses do capitalista, é mais um dos insumos necessários às suas atividades. Para o industrialista do século XIX, tratava-se principalmente de carvão; do século XX, gasolina e outros derivados do petróleo; do século XXI, eletricidade. Independente do caso, trata-se de \textit{capital circulante}, i.e. aquele que é totalmente consumido no processo produtivo.\footnote{A rigor, isso vale para o capital industrial \cite{Marx2017,Marx2014}, mas vale \textit{mutatis mutandis} para demais formas do capital (portador de juros etc.). \textbf{Certo?}}

A eletricidade, apesar de parecer algo ``etéreo'' e amorfo àquele que a emprega, também trata-se de capital circulante\footnote{Vide discussão abaixo, a eletricidade em si não é capital, mas \textit{a mercadoria cujo usufruto é energia elétrica} é, ela sim, capital (constante).}, pois, em um cenário hipotético em que ``as luzes acabem'' no instante em que a última mercadoria do dia tiver sido produzida numa dada fábrica, haverá um menor ``desperdício de capital'' do que em uma outra fábrica, idêntica à primeira, mas que tenha despendido um átimo a mais de energia na exata mesma produção.

Portanto, no que tange ao capitalista industrial, a \textit{matéria-prima}\footnote{``Quando (...) o próprio objeto do trabalho já é, por assim dizer, filtrado por um trabalho anterior, então o chamamos de matéria-prima (...) O objeto de trabalho só é matéria-prima quando já sofreu uma modificação mediada pelo trabalho.'' \cite[p. 256]{Marx2017}.} que ele emprega a fim de fornecer força-motriz a seu capital (fixo) é a substância do que ele enxerga e chama de ``energia''. Ou seja, o capitalista até pode colocar ``carvão'' em seu balancete de custos, mas ele dirá que emprega ``energia a vapor'' em sua fábrica; mais contemporaneamente, o capitalista brasileiro sabe que emprega energia gerada em hidrelétricas \textit{d'ailleurs}, mas enxerga e diz que emprega a \textit{eletricidade} que chega em sua fábrica. 


Comentar sobre Coal Question. 


1. ``The fact is, that a wasteful engine pays better where coals are cheap than a more perfect but costly engine. Bourne, in his ``Treatise on the Steam Engine,'' expressly recommends a simple and wasteful engine where coals are cheap.'' \cite[p. 6 (prefácio à segunda edição)]{Jevons1866}

2. ``It is shown that the constant tendency of discovery is to render coal a more and more efficient agent, while there is no probability that when our coal is used up any more powerful substitute will be forthcoming. Nor will the economical use of coal reduce its consumption. On the contrary, economy renders the employment of coal more profitable, and thus the present demand for coal is increased, and the advantage is more strongly thrown upon the side of those who will in the future have the cheapest supplies.'' \cite[p. 17]{Jevons1866}  --- Ótima forma de explicar Khazzoom-Brookes!!!

3. “With fuel and fire, then, almost anything is easy. By its aid in the smelting furnace or the engine we have effected, for a century past, those successive substitutions of a better for a worse, a cheaper for a dearer, a new for an old process, which advance our material civilization. But when this fuel, our material energy, fails us, whence will come the power to do equal or greater things in the future? A man cannot expect that because he has done much when in stout health and bodily vigour, he will do still more when his strength has departed. Yet such is the position of our national body, unless either the source of our strength be carefully spared, or something can be found better than coal to replace it, and carry on the substitution of the better for the worse. Whether the consumption of coal can be kept down in our free system of industry, or whether in the process of discovery we can expect to find some substitute for coal, must next be considered. The dispassionate conclusion will be far from satisfactory.” \cite[p. 72]{Jevons1866} --- Mas e quando acaba o carvão, como ficam seus prodigiosos usos?




\section{Energia enquanto usufruto de ``mercadorias-energia''}
Assim como o trabalho em si não tem valor\footnote{``No mercado, o que se contrapõe diretamente ao possuidor de dinheiro [i.e. o capitalista] não é, na realidade, o trabalho, mas o \textit{trabalhador}. O que este último vende é sua \textit{força de trabalho}. Mal seu trabalho tem início e a força de trabalho já deixou de lhe pertencer, não podendo mais, portanto, ser vendida por ele. O trabalho é a substância e a medida imanente dos valores, mas ele mesmo não tem valor nenhum.'' \cite[p. 607; grifo meu]{Marx2017}.}, a energia em si também não tem valor. O que possui valor são as mercadorias cujo usufruto provê energia --- e, no contexto de utilizabilidade humana, energia \textit{útil}. %\footnote{Ou seja, de forma que tratem-se de \textit{valores de uso}, i.e. objetos que satisfaçam necessidades humanas.} 
Por exemplo, o custo do vapor --- enquanto energia útil --- é o custo necessário para gerá-lo enquanto processo útil, um meio para fins ulteriores; idem para o custo da eletricidade, advenha ela de processos hidrelétricos, nucleares, solares, eólicos, etc.

A questão do valor da eletricidade necessariamente requer a consideração da cadeia de valor desde sua geração original (energia primária), sua transmissão e distribuição, até sua venda ao consumidor final \cite{Richter2012, PintoJr2007}. 

É neste sentido que define-se o conceito de ``eficiência energética'': é o ``custo-benefício do uso de combustíveis e energia elétrica'', i.e. do \textit{bang for your buck} que tais mercadorias trazem num processo de produção.\footnote{\cite[p. 356]{Brookes2000}. Também como Brookes o coloca no mesmo artigo, é a eficiência no que tange ao uso \textit{econômico} destas mercadorias que é relevante ao capitalista --- se tal uso é físico-quimicamente ótimo, tanto melhor!}


\section{Ineficiência energética e \textit{devil's dust}}
Fisicamente, a perda de energia devida aos processos intermediários entre a produção da energia primária e a energia útil final é algo inevitável, devido à segunda lei da Termodinâmica\footnote{``The second law of thermodynamics cannot be proved. It is believed to be valid because it leads to deductions that are in accord with observations and
experience.'' \cite[p. 100]{Wallace2006}. Ou seja, é um belo exemplo de retrodução \cite{Bhaskar1978} nas ciências naturais!}: processos termodinâmicos irreversíveis --- que são os ubiquamente observados na natureza\footnote{``The concept of reversibility is an abstraction. A reversible transformation [or ``process''] moves a system through a series of equilibrium states so that the direction of the
transformation can be reversed at any point by making
an infinitesimal change in the surroundings. \textit{All natural
transformations are irreversible to some extent}. In an \textit{irreversible} (sometimes called a \textit{spontaneous}) transformation, a system undergoes finite transformations at
finite rates, and these transformations cannot be
reversed simply by changing the surroundings of the
system by infinitesimal amounts.'' \cite[p. 100; grifo no original]{Wallace2006}. } --- fatalmente terão uma eficiência menor que $100\%$, i.e. haverá uma perda energética entre os processos\footnote{O que usualmente é descrito como ``a entropia de um processo irreversível aumenta''; ambas as asserções são equivalentes, muito embora a entropia seja uma quantidade tão efêmera à intuição cotidiana (e, em particular, à intuição de economistas).}. 

Contudo, como este ``refugo energético'' faz parte de um dado processo de produção de energia útil, sendo indissociável deste, então mesmo as mercadorias ``desperdiçadas'' neste processo são ``contabilizadas'' no valor da energia útil final. (Novamente: a energia em si não possui valor, e sim as mercadorias cujo usufruto provêm energia útil. Falar do ``valor da energia'' é meramente uma abreviação desta assertiva.) 

No que tange o processo econômico, a energia, tal qual as ideias de um empreendedor ávido por louros e louvores, não vale nada se não for \textit{efetivada} em algo objetivo. Ou seja, a energia química potencial do carvão \textit{per se} não importa, e sim sua liberação em fornos de combustão que sirvam para algo; não importa a tensão alternada da corrente elétrica que chega numa fábrica, e sim como ela traz à vida as máquinas mortas que ``empregam'' o trabalho vivo. \textbf{(Checar \cite[p. 248]{Marx2017})}

Há valor mesmo na misteriosa eletricidade que chega nos lares das máquinas industriais (e de seus leigos empregados), valor este advindo não só de sua produção original --- p. ex. na extração de carvão ou petróleo ---, como também pela depreciação envolvida em sua transmissão e conversão em energia elétrica (turbinas, cabos elétricos, transformadores, baterias, etc), além de processos auxiliares, como monitoramento do funcionamento adequado das redes elétricas, etc. 


\bibliography{mercadoria_energia.bib}



\end{document}