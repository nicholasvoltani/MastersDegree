\documentclass[12pt]{article}
\usepackage{graphicx} % Required for inserting images
\usepackage[portuguese]{babel}
\usepackage{url}
\usepackage{hyperref}

\usepackage[alf]{abntex2cite}
\bibliographystyle{abntex2-alf}

\title{Da energia enquanto mercadoria}
\author{Nicholas Funari Voltani}
\date{27 de dezembro de 2025}

\begin{document}

\maketitle

\section{}
O conceito de energia é algo que mesmo físicos do porte de Richard Feynman\footnote{Um dos ganhadores do Nobel da Física de 1965 pelas contribuições ao desenvolvimento da Eletrodinâmica Quântica (em particular pelos famosos ``diagramas de Feynman'').} consideravam ``misteriosos'', não porque ela é algo que a compreensão humana não consegue abarcar, e sim porque ela não é algo imediatamente palpável. Segundo ele,
\begin{quote}
    ``It is important to realize that[,] in physics today, we have no knowledge of what energy is. We do not have a picture that energy comes in little blobs of a definite amount. It is not that way. However, there are formulas for calculating some numerical quantity [i.e. total energy], and when we add it all together it gives [...] always the same number. It is an abstract thing in that it does not tell us the mechanism or the reasons for the various formulas.'' (\href{https://www.feynmanlectures.caltech.edu/I_04.html}{Feynman Lectures on Physics, Vol. I, Cap. 4})
\end{quote}

Para a sorte do capitalista, ele pode delegar a Física aos físicos, ocupando sua mente com o mais importante: seus negócios.\footnote{``... ele, ao contrário, é um homem prático, que nem sempre sabe o que diz quando se encontra fora do seu negócio, mas sabe muito bem o que faz dentro dele.'' \cite[p. 269]{Marx2017}.} Os mistérios do universo não importam-lhe em absoluto, importando-lhe, aí sim, o quanto tais forças naturais podem trazer-lhe ao bolso.

\section{Energia enquanto capital circulante}

Energia, no que tange aos interesses do capitalista, é mais um dos insumos necessários para seu processo de produção de mercadorias.\footnote{No que concerne ao capitalista industrial, pelo menos \cite{Marx2014}.} Para o industrialista do século XIX, tratava-se principalmente de carvão; do século XX, petróleo; do século XXI, eletricidade. 

Independente do caso, trata-se de \textit{capital circulante}, i.e. aquele que é totalmente consumido no processo produtivo \cite{Marx2017}. A eletricidade, apesar de parecer algo ``etéreo'' e amorfo àquele que a emprega, também trata-se de capital circulante\footnote{Vide discussão abaixo, a eletricidade em si não é capital, mas a mercadoria cujo usufruto é energia elétrica é, ela sim, capital (constante).}, pois, no cenário hipotético em que ``as luzes acabem'' no instante em que a última mercadoria do dia tiver sido produzida numa dada fábrica, haverá um menor ``desperdício de capital'' do que em uma fábrica, idêntica à primeira, mas que tenha despendido um átimo a mais de eletricidade.  



\section{Energia enquanto usufruto de ``mercadorias-energia''}
Assim como o trabalho em si não tem valor\footnote{``No mercado, o que se contrapõe diretamente ao possuidor de dinheiro [i.e. o capitalista] não é, na realidade, o trabalho, mas o \textit{trabalhador}. O que este último vende é sua \textit{força de trabalho}. Mal seu trabalho tem início e a força de trabalho já deixou de lhe pertencer, não podendo mais, portanto, ser vendida por ele. O trabalho é a substância e a medida imanente dos valores, mas ele mesmo não tem valor nenhum.'' \cite[p. 607; grifo meu]{Marx2017}.}, a energia em si também não tem valor. O que possui valor são as mercadorias cujo usufruto provê energia --- e, no contexto de utilizabilidade humana, energia \textit{útil}.\footnote{Ou seja, de forma que tratem-se de \textit{valores de uso}, i.e. objetos que satisfaçam necessidades humanas.} O custo do vapor --- enquanto energia útil --- é o custo necessário para gerá-lo enquanto processo útil, um meio para fins ulteriores; idem para o custo da eletricidade, advenha ela de processos hidrelétricos, nucleares, solares, eólicos, etc.

A questão do valor da eletricidade necessariamente requer a consideração da cadeia de valor desde sua geração original (energia primária), sua transmissão e distribuição, até sua venda ao consumidor final \cite{Richter2012, PintoJr2007}. 

\section{Ineficiência energética e \textit{devil's dust}}
Fisicamente, a perda de energia devida aos processos intermediários entre a produção da energia primária e a energia útil final é algo inevitável, devido à segunda lei da Termodinâmica\footnote{``The second law of thermodynamics cannot be proved. It is believed to be valid because it leads to deductions that are in accord with observations and
experience.'' \cite[p. 100]{Wallace2006}. Ou seja, é um belo exemplo de retrodução \cite{Bhaskar1978} nas ciências naturais!}: processos termodinâmicos irreversíveis --- que são os ubiquamente observados na natureza\footnote{``The concept of reversibility is an abstraction. A reversible transformation [or ``process''] moves a system through a series of equilibrium states so that the direction of the
transformation can be reversed at any point by making
an infinitesimal change in the surroundings. \textit{All natural
transformations are irreversible to some extent}. In an \textit{irreversible} (sometimes called a \textit{spontaneous}) transformation, a system undergoes finite transformations at
finite rates, and these transformations cannot be
reversed simply by changing the surroundings of the
system by infinitesimal amounts.'' \cite[p. 100; grifo no original]{Wallace2006}. } --- fatalmente terão uma eficiência menor que $100\%$, i.e. haverá uma perda energética entre os processos\footnote{O que usualmente é descrito como ``a entropia de um processo irreversível aumenta''; ambas as asserções são equivalentes, muito embora a entropia seja uma quantidade tão efêmera à intuição cotidiana (e, em particular, de economistas).}. 

Contudo, como este ``refugo energético'' faz parte de um dado processo de produção de energia útil, sendo indissociável deste, então mesmo as mercadorias ``desperdiçadas'' neste processo são ``contabilizadas'' no valor da energia útil final\footnote{Novamente: a energia em si não possui valor, e sim as mercadorias cujo usufruto provêm energia útil. Falar do ``valor da energia'' é meramente uma abreviação desta assertiva.} 

No que tange o processo econômico, a energia, tal qual as ideias de um empreendedor ávido por louros e louvores, não vale nada se não for \textit{efetivada} em algo objetivo. Ou seja, a energia potencial do carvão \textit{per se} não importa, e sim sua liberação em fornos de combustão que sirvam para algo; não importa a tensão alternada da corrente elétrica que chega numa fábrica, e sim como ela traz à vida as máquinas mortas que ``empregam'' o trabalho vivo. \textbf{(Checar \cite[p. 248]{Marx2017})}

Há valor mesmo na misteriosa eletricidade que chega nos lares das máquinas industriais (e de seus leigos empregados), valor este advindo não só de sua produção original --- p. ex. na extração de carvão ou petróleo ---, como também pela depreciação envolvida em sua transmissão e conversão em energia elétrica (turbinas, cabos elétricos, transformadores, baterias, etc), além de processos auxiliares, como monitoramento do funcionamento adequado das redes elétricas, pesquisas científicas sobre otimização de fluxos elétricos e mitigação de blecautes em cascata, etc. 


\bibliography{mercadoria_energia.bib}



\end{document}