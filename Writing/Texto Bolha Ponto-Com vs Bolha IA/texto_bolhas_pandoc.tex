% Options for packages loaded elsewhere
\PassOptionsToPackage{unicode}{hyperref}
\PassOptionsToPackage{hyphens}{url}
%
\documentclass[
]{article}
\usepackage{amsmath,amssymb}
\usepackage{iftex}
\ifPDFTeX
  \usepackage[T1]{fontenc}
  \usepackage[utf8]{inputenc}
  \usepackage{textcomp} % provide euro and other symbols
\else % if luatex or xetex
  \usepackage{unicode-math} % this also loads fontspec
  \defaultfontfeatures{Scale=MatchLowercase}
  \defaultfontfeatures[\rmfamily]{Ligatures=TeX,Scale=1}
\fi
\usepackage{lmodern}
\ifPDFTeX\else
  % xetex/luatex font selection
\fi
% Use upquote if available, for straight quotes in verbatim environments
\IfFileExists{upquote.sty}{\usepackage{upquote}}{}
\IfFileExists{microtype.sty}{% use microtype if available
  \usepackage[]{microtype}
  \UseMicrotypeSet[protrusion]{basicmath} % disable protrusion for tt fonts
}{}
\makeatletter
\@ifundefined{KOMAClassName}{% if non-KOMA class
  \IfFileExists{parskip.sty}{%
    \usepackage{parskip}
  }{% else
    \setlength{\parindent}{0pt}
    \setlength{\parskip}{6pt plus 2pt minus 1pt}}
}{% if KOMA class
  \KOMAoptions{parskip=half}}
\makeatother
\usepackage{xcolor}
\setlength{\emergencystretch}{3em} % prevent overfull lines
\providecommand{\tightlist}{%
  \setlength{\itemsep}{0pt}\setlength{\parskip}{0pt}}
\setcounter{secnumdepth}{-\maxdimen} % remove section numbering
\newlength{\cslhangindent}
\setlength{\cslhangindent}{1.5em}
\newlength{\csllabelwidth}
\setlength{\csllabelwidth}{3em}
\newlength{\cslentryspacingunit} % times entry-spacing
\setlength{\cslentryspacingunit}{\parskip}
\newenvironment{CSLReferences}[2] % #1 hanging-ident, #2 entry spacing
 {% don't indent paragraphs
  \setlength{\parindent}{0pt}
  % turn on hanging indent if param 1 is 1
  \ifodd #1
  \let\oldpar\par
  \def\par{\hangindent=\cslhangindent\oldpar}
  \fi
  % set entry spacing
  \setlength{\parskip}{#2\cslentryspacingunit}
 }%
 {}
\usepackage{calc}
\newcommand{\CSLBlock}[1]{#1\hfill\break}
\newcommand{\CSLLeftMargin}[1]{\parbox[t]{\csllabelwidth}{#1}}
\newcommand{\CSLRightInline}[1]{\parbox[t]{\linewidth - \csllabelwidth}{#1}\break}
\newcommand{\CSLIndent}[1]{\hspace{\cslhangindent}#1}
\ifLuaTeX
  \usepackage{selnolig}  % disable illegal ligatures
\fi
\IfFileExists{bookmark.sty}{\usepackage{bookmark}}{\usepackage{hyperref}}
\IfFileExists{xurl.sty}{\usepackage{xurl}}{} % add URL line breaks if available
\urlstyle{same}
\hypersetup{
  pdftitle={Ao vencedor, as batatas: análise das crises econômicas enquanto reposições da acumulação do capital},
  pdfauthor={Nicholas Funari Voltani},
  hidelinks,
  pdfcreator={LaTeX via pandoc}}

\title{Ao vencedor, as batatas: análise das crises econômicas enquanto reposições da acumulação do capital}
\author{Nicholas Funari Voltani}
\date{28 de janeiro de 2025}

\begin{document}
\maketitle

\emph{And on the pedestal these words appear:}

``My name is Ozymandias, king of kings:

Look on my works, ye mighty, and despair!''

Nothing beside remains: round the decay

Of that colossal wreck, boundless and bare,

The lone and level sands stretch far away.

Percy Bysshe Shelley (1818)

A noção de \emph{ciclos econômicos} --- ou seja, da atividade econômica enquanto algo inerentemente cíclico --- não é nova, sendo objeto da economia política já desde, no mínimo, o século XVIII. Um ciclo econômico é, em sua essência, um movimento \emph{periódico} da economia (geralmente aferido pelas flutuações do PIB), ou seja, com períodos de picos e vales, os quais sucedem-se uns aos outros: um movimento de ``ascensão'' leva um vale a um pico, e um movimento de ``queda'' leva um pico a um vale. Ao primeiro pode-se dar o nome de ``prosperidade'' e, ao último, de ``recessão''.

Pela ótica marxista, em particular, uma fase de prosperidade condiz com uma fase de acumulação de capital, e uma fase de recessão, com uma fase de destruição de capital. Tais movimentos, porém, não restringem-se à produção/destruição \emph{física} de valores de uso, e sim de produção/destruição de capital \emph{lato sensu}, o qual não restringe-se à sua forma física de máquinas e mercadorias. Mais que isso, a História possui alguns tantos casos em que a acumulação e produção exacerbada de capital \emph{fictício} deixou carcassas materiais de capital real inativo --- ou seja, capital ``mais-que-morto'', sem emprego algum, ``não-capitais''.\footnote{Capital somente é capital porquanto está \emph{em movimento}; quando não está \emph{in motu}, ou seja, quando não é valor que se valoriza, pode até ser valor, mas não pode ser \emph{capital}. Ao menos no que tange ao capital industrial (``capital real''), vale que ``O ciclo do capital só se desenrola normalmente enquanto suas distintas fases se sucedem sem interrupção. Se o capital estaciona na segunda fase D-M, o capital {[}dinheiro{]} se enrijece como tesouro; se estaciona na fase de produção, tem-se, de um lado, que os meios de produção restam desprovidos de qualquer função e, de outro, que a força de trabalho permanece ociosa; se estaciona na última fase M'-D', as mercadorias não vendidas e acumuladas bloqueiam o fluxo da circulação.'' (Marx, 2014).}

\hypertarget{o-cenuxe1rio-puxf3s-2008-e-a-duxe9cada-do-machine-learning}{%
\section{\texorpdfstring{O cenário pós-2008, e a década do \emph{machine learning}}{O cenário pós-2008, e a década do machine learning}}\label{o-cenuxe1rio-puxf3s-2008-e-a-duxe9cada-do-machine-learning}}

Após o cataclisma financeiro de 2008 e a crise global que o sucedeu, o capital buscou novos Santos Graais ao longo dos anos 2010, novos caminhos de autovalorização.

Muitos buscaram tornar-se sua nova menina dos olhos. \emph{Blockchains}, por exemplo, eram algo ainda muito abstrato e alheio ao interesse centralizante e hierárquico do capital, e brilhavam menos aos olhos do que as cada vez mais faladas criptomoedas. Nem tudo que brilha é ouro, porém, e, hoje em dia, o mercado de cripto prolifera sumamente em fóruns de entusiastas e aficionados, transações legalmente duvidosas, e em discursos ainda mais duvidosos de certos patetas presidentes\footnote{E em seus bolsos também. Javier Milei, contudo, disse que nada teve a ver com o escândalo da criptomoeda \$LIBRA em 2025, após ter publicado no Twitter para que a comprassem, naturalmente sendo, por sua formação, alheio à algaravia do mundo econômico. Quando questionado em entrevista se ajudou a divulgar a criptomoeda através de sua conta pessoal, respondeu que ``\emph{yo no lo promocioné, ¡yo lo difundí!}'', e lá retificou-se, dizendo, abre aspas, ``\emph{A las ratas inmundas de la casta política que quieren aprovechar esta situación para hacer daño{[},{]} les quiero decir que todos los días confirman lo rastreros que son los políticos}'' (contando ele próprio também?), ``\emph{y aumentan nuestra convicción de sacarlos a patadas en el culo.}'' (Contando ele próprio também?) Donald Trump, por outro lado, certamente lançou sua criptomoeda \$TRUMP (mui modesta alcunha) dias antes de iniciar seu segundo mandato presidencial por motivos nobres. O resultado de ambos, todos sabem: um pico de demanda pelas moedas, enorme inflação de seus preços, venda massiva delas por \emph{insiders}, e uma multidão de pobres coitados com as mãos abanando.}.

Por um tempo, a computação quântica, com seu nome bombástico --- um \emph{sine qua non} para o corporativo --- e com um quê de misterioso e científico --- o que é redundante para o corporativo ---, não cumpriu sua promessa de trazer um \emph{Minimum Viable Product} em tempo hábil, e logo caiu em desgraça.\footnote{Em um relatório de abril de 2024 chamado ``\emph{Steady progress in approaching the quantum advantage}'', McKinsey \& Company mostra que o \emph{start-up investment} em Computação Quântica havia caído em \(27\%\) entre 2022 e 2023 (de US\$ 2,35 bilhões para US\$ 1,71 bilhões), com os crescentes investimentos em IA generativa certamente desempenhando um papel nessa fuga de capital. Não os impediu de publicar, em junho de 2025, o relatório ``\emph{The Year of Quantum: From concept to reality in 2025}'', em que dizem que ``\emph{While QT {[}Quantum Technologies{]} will affect many industries, the chemicals, life sciences, finance, and mobility industries will see the most growth.}'' --- mas, infelizmente, já não havia mais plateia para ouvi-los, e quiçá nem mesmo seres humanos que tenham escrito o próprio relatório.} Vide publicação de Sankar Das Sarma (da própria área de computação quântica, com mais de 130.000 citações pelo Google Scholar): ``\emph{Quantum mechanics is indeed weird and counterintuitive, but that by itself does not guarantee revenue and profit.}'' (Das Sarma, 2022).\footnote{Isso, claro, não impediu charlatões de venderem óleos de serpente como ``transformação quântica'' (como a \emph{saudosíssima} Quantchums, uma empresa de fundo de quintal, obliterada pelo colapso de alguma função de onda) para inocentes \emph{managers} corporativos em busca de diferenciar seus CVs. \emph{Hell if I know!}}

E, por fim, a IA. A década de 2010 marcou o \emph{revival} da Inteligência Artificial nos meios acadêmicos, transbordando até mesmo para fora dele, facilmente podendo ser condecorada como a década de \emph{machine learning}. A área de redes neurais profundas (\emph{deep neural networks}) e o desenvolvimento de novas espécies destas --- destacam-se principalmente redes neurais convolucionais (\emph{convolutional neural networks}) e suas subsequentes variantes\footnote{Muito desses \emph{breakthroughs} saíram das competições Imagenet (\emph{ImageNet Large Scale Visual Recognition Challenge}, 2010-2017) de reconhecimento visual ``de máquina'', por exemplo a famosa Alexnet (Krizhevsky; Sutskever; Hinton, 2012) --- aliás, já um exemplo de uso de GPUs para treinamento de redes neurais (no caso, convolucionais) mais de 10 anos atrás! (Inclusive, GPUs da Nvidia, ``\emph{two {[}Nvidia Geforce{]} GTX 580 3GB GPUs}'', Ibid. p. 2.)}, e redes neurais recorrentes (\emph{recurrent neural networks}) \emph{ditto}\footnote{Para referência, os exemplos mais dignos de nota certamente são as redes Long-Short Term Memory (LSTM) (Hochreiter; Schmidhuber, 1997) (um paper de 1997, somente aproveitado 20 anos depois!), e as redes Transformers, propostas por (Vaswani et al., 2017), cruciais para a compreensão do \emph{boom} da IA nos 2020's. O reinado no imaginário popular das redes Transformer já tinha sido prenunciado, por exemplo, por Leo Dirac em sua palestra de 2019, ``\emph{LSTM is dead. Long Live Transformers!}'' (Dirac, 2019): o famoso modelo ChatGPT, por exemplo, trata-se justamente de um modelo ``\emph{Generative Pre-Trained \textbf{Transformer}}''!} --- fincou suas raízes no \emph{Zeitgeist}, e, mais importante para o capitalista, nos negócios.

Esta área da tecnologia, que tanto está na boca do povo hoje, já havia encontrado guarita nos meios corporativos ao longo dos anos 2010, porém mais esparsamente. Sob a forma de modelos de regressão (não-)linear e de árvores de decisão\footnote{Que são as aplicações mais básicas da área de \emph{machine learning}. Dentre as referências introdutórias mais famosas da área, destacam-se os livros \emph{Neural Networks and Deep Learning} (Nielsen, 2015) e \emph{Deep Learning} (Goodfellow; Bengio; Courville, 2016).}, o poder das redes neurais só era devidamente conjurado mediante quantidades substanciais de dados (i.e. o assim-chamado ``\emph{Big Data}''), e de poder computacional suficiente para usufrui-la.\footnote{``\emph{Availability of digital information (data) and computing resources (in this case, in the form of cloud computing) appear in our analysis to be central to AI adoption across a very broad swath of firms.}'' (McElheran et al., 2023, p. 28).} Ou seja, nenhuma ``dor de dono'' ou \emph{growth mindset} aplacaria a insaciável fome de que redes neurais padecem, não de sonhos e ambições, e sim de eletricidade e dados bem-formatados.

No artigo de (McElheran et al., 2023) pelo \emph{National Bureau of Economic Research}, analisa-se um censo feito com uma amostra de mais de 850.000 firmas sobre seu uso de IA\footnote{A definição de IA que foi empregada neste censo é crucial: ``\emph{Specific definitions are provided (...) and include five commonly associated with AI (bolded): machine learning, machine vision, natural language processing, voice recognition software, and automated guided vehicles.}'' (McElheran et al., 2023, p. 13). Note-se que perguntam não meramente sobre ``uso de IA'' \emph{hand-wavingly}, e sim ``uso de \emph{machine learning}/\emph{natural language processing} etc.'', i.e. as perguntas foram bem formuladas, não deixaram margem para interpretações leigas dessa palavra etérea ``IA''.}, e seus resultados são que:

\begin{quote}
``\emph{Our representative statistics indicate that \textbf{just under 6\% of firms nationwide used AI as of 2017}.''} (O \emph{paper} é de outubro de 2023.) ``\emph{Yet \textbf{most very large firms (over 5,000 employees) reported at least some AI use}, leading to employment-weighted adoption of 18\%. Intensity varied from merely testing (1.1\%) to using AI in more than a quarter of production (2.2\%). More-intensive use was prevalent among 25-30\% of the largest firms, attesting to its skewness.} (McElheran et al., 2023, p. 2; grifo meu)''
\end{quote}

Tais tecnologias, infelizmente para o capitalista, não cumpriram a promessa de completa liberação de força de trabalho que seus entusiastas tanto alardeavam. Não bastasse o tempo que se levava para treinar uma rede neural com grande quantidade de parâmetros e sobre uma grande quantidade de dados (e o tempo de retreiná-la, especialmente após ocasionais erros...), os modelos recém-treinados dificilmente eram interpretáveis --- quem dirá auditáveis, caso dessem errado! Infelizmente, estes modelos precisavam explicar-se a seus superiores através de bocas humanas, e, caso não se explicassem suficientemente bem, requereriam dores de cabeça também humanas para interpretar sua viabilidade.\footnote{Até existe a área de \emph{explainable AI/Machine Learning} (xAI/xML), mas trata-se uma área auxiliar, e não tão grande, dentro do enorme panorama de \emph{machine learning}. De qualquer forma, também estes modelos não têm boca para declararem-se inocentes frente a possíveis erros que venham a cometer.}

Para variar, isso não impediu que esses modelos fossem vendidos para públicos aos quais eles não eram apropriados. E, como sempre, o medo-mor do capitalista é de ficar para trás de quem aproveita oportunidades que ele próprio não aproveitou. Mas, no fim e ao cabo, embora todas as firmas tivessem as mesmas 24 horas no dia, elas certamente não tinham as mesmas quantidades suficientes de dados para usufruir adequadamente de seus novos modelos. \emph{Vae soli! Vae victis!}

\hypertarget{a-primeira-metade-dos-anos-2020-e-a-bolha-de-ia}{%
\section{A primeira metade dos anos 2020, e a bolha de IA}\label{a-primeira-metade-dos-anos-2020-e-a-bolha-de-ia}}

Enquanto cientistas da computação comemoraram uma década de revoluções nos anos 2010, a primeira metade de 2020 presenciou, em particular, um \emph{breakthrough} na capacidade de redes neurais ``Transformers'', responsáveis por processamento de linguagem natural (\emph{natural language processing, NLP}). A diferença agora foi que a novidade foi mais fácil de vender ao público: é difícil vender para um leigo um modelo que é capaz de transformar palavras em vetores de espaços vetoriais multidimensionais etc. etc., mas quem não ficaria empolgado de escrever uma pergunta para um robô e receber uma resposta que remotamente pareça humana?\footnote{A propósito, estes modelos dependem, de fato, dessas representações de palavras em tais espaços vetoriais, método este chamado de ``\emph{word embedding}''. Uma interessante leitura sobre este método --- e sobre a reprodução de estereótipos humanos nestes modelos --- é o artigo ``\emph{Man is to computer programmer as woman is to homemaker? Debiasing word embeddings}'' de (Bolukbasi et al., 2016).} E, dessa forma, mais e mais Pinóquios digitais começaram a proliferar pelo mercado, e algumas pessoas até juraram de pé junto que eles eram humanos.\footnote{Vide e.g. ``\emph{Google engineer put on leave after saying AI chatbot has become sentient}'', The Guardian, 12 de junho de 2022. (Não ajuda que o perfil no Medium deste engenheiro possui a descrição: ``\emph{I'm a software engineer. I'm a priest. I'm a father. I'm a veteran. I'm an ex-convict. I'm an AI researcher. I'm a cajun. I'm whatever I need to be next.}''...)}

No que tange à presente discussão, não importa tanto o que os novos modelos LLM (\emph{Large Language Models}) fazem; na verdade, nos importa mais o que eles \emph{não} fazem: trazer o exorbitante rendimento que está sendo alardeado no mercado ao longo dos últimos anos.

De fato, há um abismo entre as somas de dinheiro investido em inteligência artificial atualmente versus suas margens de lucro. Que seja dado nomes a alguns bois, portanto.

OpenAI estipula US\$ 1 trilhão para seu IPO em 2026/2027 (Wang et al., 2025), e estima que terá uma receita de US\$ 100 bilhões em 2029 --- quase a mesma receita que a Nvidia (US\$ 130 bilhões), uma empresa bem consolidada e o monopólio fulcral da bolha de IA, conseguiu em 2025 (Laird, 2026) --- enquanto estima começar a ter lucros (ou seja, sobrepujar suas perdas) somente em... 2029 (Smith, 2026)! Uma empresa, que sequer possui IPO e estima estar no negativo até 2029, mira o rendimento da atual maior empresa do mundo até 2029! Que \emph{growth mindset}!

CoreWeave, uma empresa central neste meio e que aluga o uso computacional de \emph{data centers} para IA, simplesmente teve uma queda de 6\% em suas ações no mesmo dia da publicação de uma reportagem do The Verge (Lopatto, 2025) que alegava que a empresa ``está no coração da bolha de IA'', totalmente emaranhada em dívidas, muitas das quais colateralizadas em GPUs da Nvidia, emitidas pela... Nvidia, e daí para morro abaixo (Juricic, 2025).

E não menos importante: Nvidia, o alicerce material que sustenta todo o \emph{hype} de IA do momento. Através de seu modelo CUDA de programação paralela (já um grande conhecido da área desde antes do atual \emph{boom} da IA), o qual é atrelado ao \emph{hardware} (GPUs) que ela produz, a Nvidia detém um monopólio sobre o treinamento de modelos LLM em \emph{data centers}. Dessa forma, a empresa surfou a onda atual e se tornou a primeira empresa a deter US\$ 5 trilhões de valor de mercado (Kaye, 2025), e tornou-se a maior empresa do mundo --- e com um valor maior do que o PIB de todos os países do mundo, exceto EUA, China e Alemanha.

O treinamento de modelos de IA, devido à enorme quantidade de operações necessárias para a otimização de seus parâmetros, requer muito poder computacional e, sobretudo, energia, e tudo isso é potencializado quanto mais dados de treinamento são empregados no processo. Não é à toa que foi somente com a proliferação de modelos GPT --- \emph{Generative Pre-Trained Transformers} --- que modelos de \emph{machine learning} se popularizaram para além de seus nichos e círculos de (bem aparelhados)\emph{early adopters}: porque agora esses custos são ocultados, expandindo sua atratividade para possíveis usuários finais e, quiçá, clientes.

Em meio a tudo isso, os negócios do setor de IA têm ido de vento em popa, crescimentos do PIB, \emph{numbers go up}, \emph{Friede, Freude, Eierkuchen}... O que pode dar errado?

\hypertarget{superproduuxe7uxe3o-de-capital-fictuxedcio-e-a-crise}{%
\section{Superprodução de capital (fictício) e a crise}\label{superproduuxe7uxe3o-de-capital-fictuxedcio-e-a-crise}}

No que tange o fator especulativo do atual \emph{hype} da IA, um causo que mostrou que o rei estava nu foi a divulgação, em janeiro de 2025, do modelo R1 da empresa chinesa DeepSeek, o qual custou cerca de US\$ 5 milhões para ser desenvolvido --- de 2 até 4 ordens de magnitude menos que os modelos ocidentais! Resultado: US\$ 1 \emph{trilhão} (com T) apagado do mapa, dentro de 24 horas, dentre os quais: Alphabet perdeu US\$ 100 bilhões de valor de mercado; Microsoft, US\$ 7 bilhões; e, naturalmente, Nvidia teve mais a perder, US\$ 600 bilhões (OpenAI não possuía IPO) (Milmo et al., 2025). Tudo isso sem a destruição de um \emph{data center} sequer.

Os atuais movimentos especulativos do mercado de IA são uma ótima ilustração do conceito de capital fictício, que consiste em um título de direito a certa remuneração, o qual pode ser comercializado e cujo preço pode autonomizar-se da fonte dessa remuneração (Marx, 2017a). O caso mais exemplar de capital fictício, além de títulos da dívida pública, são as ações de uma empresa. Tenha-se, como exemplo, uma empresa produtiva cujas ações sejam originalmente atreladas \emph{pro rata} ao seu capital total; então, a variação de preço de uma ação induzirá uma variação do capital fictício existente, assumindo que o capital total se mantenha constante. Há, portanto, uma autonomização entre o conteúdo da ação --- uma parcela daquele capital total e, mais importante, o direito a uma parcela de seus rendimentos --- e o preço de mercado dessa ação.\footnote{(Carcanholo; Sabadini, 2009) o chamam de ``capital fictício de tipo 2'', equivalente à categoria que nomearam ``lucros fictícios'' --- ou seja, criação/destruição (na totalidade) de capital fictício.}

Assim como o capital portador de juros, o capital fictício tem seu preço dependente da oferta e demanda de títulos de apropriação a rendimentos, naturalmente desatrelando-se do valor do capital a que ele remete --- seja este capital real ou, ele próprio, fictício\footnote{Em cujo caso não haveria capital real subjacente e nem, portanto, valor; haveria, sim, um \emph{preço} atrelado a este capital fictício subjacente, o máximo de ``valor'' que um capital fictício é capaz de possuir.}. É justamente a possibilidade de que tais títulos não apontem a algum capital real o que distingue o capital fictício do capital portador de juros, este último dependendo de que se empreste capital-dinheiro a algum capital industrial a fim de obter juros, i.e. uma parcela do mais-valor que este venha a produzir. O capital fictício, ao contrário, pede que ``façam o que ele diz, e não o que ele {[}não{]} faz''.

Não obstante tendo um alicerce transparente --- haja algo real ali ou não ---, o capital fictício \emph{aparece} como algo real, e é efetivamente capaz de fornecer poder de compra real a seu detentor, caso ele venda-o --- afinal, dinheiro é dinheiro para qualquer um.\footnote{``Como a mercadoria desaparece ao se transformar em dinheiro, neste não se percebe como ele chegou às mãos de seu possuidor ou qual mercadoria foi nele transformada. O dinheiro \emph{non olet} {[}não fede{]}, seja qual for sua origem. Se{[},{]} por um lado{[},{]} ele representa mercadoria vendida, por outro representa mercadorias compráveis.'' (Marx, 2017b, p. 184).} Aqueles que permanecem com capital fictício em mãos, porém, tacitamente alegam saber que nestes títulos se lê, em letras capitais: \emph{hic sunt dracones}.

É neste sentido que a atual situação da indústria de IA é uma bolha: porque os maravilhosos números dos \emph{Magnificent 7}\footnote{Alphabet, Amazon, Apple, Tesla, Meta, Microsoft e Nvidia.} estão enormemente desatrelados à produção de valor real, cujos preços de ações são muito mais dependentes de expectativas sobre o desempenho futuro de suas respectivas empresas do que de seus desempenhos presentes --- e, neste quesito, a OpenAI é talvez o maior exemplar do panorama atual.

O objetivo dos especuladores --- e hoje é inegável que estejamos em uma bolha especulativa ---, no entanto, é bem sabido e já foi muito bem colocado por Keynes há quase um século:

\begin{quote}
``\emph{The social object of skilled investment should be to defeat the dark forces of time and ignorance which envelop our future. The actual, private object of the most skilled investment to-day is `to beat the gun', as the Americans so well express it}, to outwit the crowd, and to pass the bad, or depreciating, half-crown to the other fellow.'' (Keynes, 2013, p. 155; grifo meu)
\end{quote}

\hypertarget{conclusuxf5es}{%
\section{Conclusões}\label{conclusuxf5es}}

A situação atual da euforia da IA tem sido comparada com a bolha das ponto-com, aqui e ali, nos últimos anos. Um bom resumo do motivo:

\begin{quote}
``\emph{Cisco was pitched as the crucial infrastructure provider for the internet revolution --- whatever the internet became, we were told at the end of the last century, it would use Cisco switches and routers to become it. Nvidia is pitched as the crucial infrastructure provider for the artificial intelligence revolution --- whatever AI becomes, we are told, it will use Nvidia chips to become it.}'' (Armstrong, 2023)
\end{quote}

O que a Nvidia é hoje para a atual bolha de IA, a Cisco foi para a bolha das ponto-com, em que ambos são os ``\emph{``picks and shovels'' of a digital gold rush}''(Kollar, 2025), i.e. ambos forneceram a estrutura material sobre a qual toda a especulação no mundo digital assenta-se --- especialmente quando assenta-se sobre a ``imaterial'' nuvem.\footnote{``\emph{But just as the clouds above us, however formless or ethereal they may appear to be, are in fact made of matter---water molecules in various states of condensation and crystallization---the Cloud of the digital is also relentlessly material.}'' (Monserrate, 2022, p. 3). Um excelente artigo --- antropológico! --- sobre infraestrutura digital.}

Contudo, é importante reconhecer também que o cenário em que tais bolhas acontecem é influenciado pelo panorama político. A bolha das ponto-com deu-se num contexto de unipolarização do mundo ao redor dos Estados Unidos, e sob os auspícios do \emph{Telecommunications Act} de 1996, que visava ``desregular'' o mercado e viabilizar a indústria de telecomunicações --- na prática, deu rédeas ao livre-mercado doméstico e ao \emph{homo homini lupus} da concorrência capitalista desta indústria \footnote{``\emph{The Telecommunications Act of 1996, the first major overhaul of U.S. communications law in six decades, was marketed as deregulation but functioned as industrial policy. By privatizing the internet backbone and dismantling barriers between local and long-distance carriers, phone and cable companies, and telecom and internet services, it rewrote the rules to channel capital into fiber, switching, and long-haul infrastructure. The Act, presented as serving ``the public interest,'' opened new markets and spurred unprecedented private investment. Yet rather than fostering lasting competition, as its boosters claimed, it accelerated consolidation. Mergers among major telecommunications firms soon concentrated control of the internet's backbone in the hands of a few corporations---an outcome industry insiders had anticipated.}''(Kollar, 2025)}. A atual bolha da IA ocorre sob a vigência do \emph{CHIPS and Science Act} de 2022 e a guerra tarifária de Trump desde 2025, que funcionam efetivamente como políticas industriais defensivas frente à hegemonia chinesa, tomando mesmo o caráter de políticas de segurança nacional.\footnote{``\emph{Nvidia's success, by contrast, is bound up with defensive industrial policy aimed at maintaining a lead over China. The CHIPS and Science Act channels billions into semiconductor research and fabrication, while export controls restrict the sale of advanced GPUs abroad, effectively enrolling Nvidia into U.S. national security strategy. (Kollar, 2025)}''}

Como Kollar o resume,

\begin{quote}
``\emph{Where Cisco once rode a wave of \textbf{liberalization to global expansion}, Nvidia advances in a climate of \textbf{securitized competition}, with industrial policy aimed less at opening markets than at securing them.}'' (Kollar, 2025, grifo meu)
\end{quote}

Quando da bolha das ponto-com, houve muito capital fixo mobilizado para sua viabilização, o qual permaneceu após a crise --- a chamada \emph{dark fiber} --- e que foi avidamente empregada em ímpetos posteriores do capital --- dentre os quais seu atual ímpeto. Como também bem colocado por Kollar,

\begin{quote}
``\emph{The glut of idle `dark fiber' from the crash eventually lowered bandwidth costs and supported the growth of broadband and the cloud. Cisco's trajectory revealed how speculative booms can destroy financial capital while leaving lasting material legacies and a powerful firm {[}i.e. Cisco{]} intact.}'' (Kollar, 2025)
\end{quote}

O mesmo pode ser dito da \emph{railway mania} do século XIX, que foi liberalmente sancionada pelo Parlamento britânico, como se pode constatar, por exemplo, pelos 263 (!) \emph{Acts of Parliament} que foram passados \emph{no ano de 1846} (Roberts, 2025), e que inegavelmente mobilizou capital fixo que viabilizou acumulações posteriores de capital.

\begin{quote}
``\emph{After all, even though the 1873 panic led to a market collapse in railway stocks and a deep recession, in the end, the US had a legacy of a rail network across the continent. Similarly in Britain, after the slump of the late 1840s, the subsequent long boom of the 1850s rested partly on the 6,000-mile rail network which then formed the backbone of the country's transportation system and helped Britain maintain its global hegemony.}'' (Roberts, 2025)
\end{quote}

Analogamente, haverão muitos \emph{data centers} parcial ou totalmente inoperantes pós-crash da atual bolha de IA, o que fará o valor deste capital cair vertiginosamente. Em certo momento, este capital voltará a exercer atração a novas empreitadas --- e, como ele já está mobilizado e ``pronto para uso'', seria um ``\emph{desperdício}'' não empregá-lo.\footnote{Há, porém, o problema da depreciação, tanto do capital ocioso, quanto a depreciação moral das GPUs que ficarão \emph{outdated} com o tempo, que também deve ser tomado em mente.}

O resumo da ópera, porém, sempre é que, envolto pelo turbilhão de uma crise, o capitalista perde a cupidez com que enxergava sua propriedade --- seja ela física ou de \emph{papers}. Não consegue olhar senão com desilusão para o passado que o trouxe ao presente desfiladeiro, e com desespero para a queda contra a qual cambaleia --- alguns mais próximos do abismo, outros menos. Nada que impeça, não obstante, que alguns abram suas garrafas de espumante, e que suas rolhas caiam no abismo junto dos valores de capitais frescos no mercado.

\hypertarget{refs}{}
\begin{CSLReferences}{0}{1}
\leavevmode\vadjust pre{\hypertarget{ref-Armstrong2023}{}}%
ARMSTRONG, Robert. Nvidia circa 2023, {Cisco} circa 2000. \textbf{Financial Times}, ago. 2023.

\leavevmode\vadjust pre{\hypertarget{ref-Bolukbasi2016a}{}}%
BOLUKBASI, Tolga \emph{et al.} \href{https://proceedings.neurips.cc/paper_files/paper/2016/hash/a486cd07e4ac3d270571622f4f316ec5-Abstract.html}{Man Is to Computer Programmer as Woman Is to Homemaker? {Debiasing} Word Embeddings}. \textbf{Advances in neural information processing systems}, v. 29, 2016.

\leavevmode\vadjust pre{\hypertarget{ref-Carcanholo2009}{}}%
CARCANHOLO, Reinaldo; SABADINI, Mauricio. Capital Fictício e Lucros Fictícios. \textbf{Revista da Sociedade Brasileira de Economia Política}, v. 24, n. 41, p. 65, 2009.

\leavevmode\vadjust pre{\hypertarget{ref-DasSarma2022}{}}%
DAS SARMA, Sankar. \textbf{Quantum Computing Has a Hype Problem}. \textbf{MIT Technology Review}, mar. 2022. Disponível em: \textless{}\url{https://www.technologyreview.com/2022/03/28/1048355/quantum-computing-has-a-hype-problem/}\textgreater. Acesso em: 25 jan. 2026

\leavevmode\vadjust pre{\hypertarget{ref-Dirac2019}{}}%
DIRAC, Leo. \textbf{{LSTM} Is Dead. {Long Live Transformers}!} Seattle Applied Deep Learning, dez. 2019. Disponível em: \textless{}\url{https://www.youtube.com/watch?v=S27pHKBEp30}\textgreater. Acesso em: 23 jan. 2026

\leavevmode\vadjust pre{\hypertarget{ref-Goodfellow-et-al-2016}{}}%
GOODFELLOW, Ian; BENGIO, Yoshua; COURVILLE, Aaron. \textbf{\href{https://www.deeplearningbook.org/}{Deep Learning}}. \emph{{[}S.l.{]}}: MIT Press, 2016.

\leavevmode\vadjust pre{\hypertarget{ref-Hochreiter1997}{}}%
HOCHREITER, Sepp; SCHMIDHUBER, Jürgen. \href{https://ieeexplore.ieee.org/abstract/document/6795963/}{Long Short-Term Memory}. \textbf{Neural computation}, v. 9, n. 8, p. 1735--1780, 1997.

\leavevmode\vadjust pre{\hypertarget{ref-Juricic2025}{}}%
JURICIC, Louis. \textbf{{CoreWeave} Stock Falls after {The Verge} Labels It "Heart of {AI} Bubble"}. \textbf{Yahoo Finance}, nov. 2025. Disponível em: \textless{}\url{https://finance.yahoo.com/news/coreweave-stock-falls-verge-labels-155210921.html}\textgreater. Acesso em: 26 jan. 2026

\leavevmode\vadjust pre{\hypertarget{ref-Kaye2025}{}}%
KAYE, Danielle. \textbf{Nvidia Becomes World's First \$5tn Company}., out. 2025. Disponível em: \textless{}\url{https://www.bbc.com/news/articles/cp8e970vn5vo}\textgreater. Acesso em: 27 jan. 2026

\leavevmode\vadjust pre{\hypertarget{ref-Keynes2013}{}}%
KEYNES, John Maynard. \textbf{The Collected Writings of {John Maynard Keynes}: {The General Theory} of {Employment}, {Interest} and {Money}}. Cambridge: Cambridge University Press for the Royal Economic Society, 2013. v. 7

\leavevmode\vadjust pre{\hypertarget{ref-Kollar2025}{}}%
KOLLAR, Justin. \textbf{Cisco, {Nvidia}, and the {Economics} of {Tech Bubbles}}., nov. 2025. Disponível em: \textless{}\url{https://www.technostatecraft.com/p/cisco-nvidia-and-the-economics-of?utm_medium=android\&triedRedirect=true}\textgreater. Acesso em: 22 dez. 2025

\leavevmode\vadjust pre{\hypertarget{ref-Krizhevsky2012}{}}%
KRIZHEVSKY, Alex; SUTSKEVER, Ilya; HINTON, Geoffrey E. \href{https://doi.org/10.1145/3065386}{{ImageNet} Classification with Deep Convolutional Neural Networks}. \textbf{Communications of the ACM}, v. 60, n. 6, p. 84--90, 2012.

\leavevmode\vadjust pre{\hypertarget{ref-Laird2026}{}}%
LAIRD, Jeremy. \textbf{{OpenAI}'s Own Forecast Predicts \$14 Billion Loss in 2026 but {Nvidia-style} \$100 Billion Revenues by 2029 According to New Report}. \textbf{Yahoo Finance}, jan. 2026. Disponível em: \textless{}\url{https://www.pcgamer.com/software/ai/openais-internal-documents-predict-usd14-billion-loss-in-2026-according-to-report/}\textgreater. Acesso em: 27 jan. 2026

\leavevmode\vadjust pre{\hypertarget{ref-Lopatto2025}{}}%
LOPATTO, Elizabeth. \href{https://www.theverge.com/ai-artificial-intelligence/822011/coreweave-debt-data-center-ai}{Meet {CoreWeave}, the {AI} Industry's Ticking Time Bomb}. \textbf{The Verge}, nov. 2025.

\leavevmode\vadjust pre{\hypertarget{ref-Marx2014}{}}%
MARX, Karl. \textbf{{O capital: crítica da economia política - Livro II: o processo de circulação do capital}}. 1. ed. São Paulo: Boitempo, 2014.

\leavevmode\vadjust pre{\hypertarget{ref-Marx2017}{}}%
MARX, Karl. \textbf{{O Capital: crítica da economia política - Livro I: o processo de produção do capital}}. 2. ed. São Paulo: Boitempo, 2017b.

\leavevmode\vadjust pre{\hypertarget{ref-Marx2017a}{}}%
MARX, Karl. \textbf{{O capital: crítica da economia política - Livro III: o processo global da produção capitalista}}. 1. ed. São Paulo: Boitempo, 2017a.

\leavevmode\vadjust pre{\hypertarget{ref-McElheran2023}{}}%
MCELHERAN, Kristina \emph{et al.} {AI} Adoption in {America}: {Who}, What, and Where. \textbf{NBER Working Paper Series}, 2023.

\leavevmode\vadjust pre{\hypertarget{ref-Milmo2025}{}}%
MILMO, Dan \emph{et al.} \href{https://www.theguardian.com/business/2025/jan/27/tech-shares-asia-europe-fall-china-ai-deepseek}{{``{Sputnik} Moment''}: \$1tn Wiped off {US} Stocks after {Chinese} Firm Unveils {AI} Chatbot}. \textbf{The Guardian}, jan. 2025.

\leavevmode\vadjust pre{\hypertarget{ref-Monserrate2022}{}}%
MONSERRATE, Steven Gonzalez. \href{https://mit-serc.pubpub.org/pub/the-cloud-is-material/release/1}{The Cloud Is Material: {On} the Environmental Impacts of Computation and Data Storage}. 2022.

\leavevmode\vadjust pre{\hypertarget{ref-Nielsen2015}{}}%
NIELSEN, Michael A. \textbf{\href{http://neuralnetworksanddeeplearning.com/}{Neural Networks and Deep Learning}}. San Francisco: Determination Press, 2015. v. 25

\leavevmode\vadjust pre{\hypertarget{ref-Roberts2025}{}}%
ROBERTS, Michael. \textbf{{AI} and the Railway Mania}. \textbf{Michael Roberts Blog}, dez. 2025. Disponível em: \textless{}\url{https://thenextrecession.wordpress.com/2025/12/04/ai-and-the-railway-mania/}\textgreater. Acesso em: 23 dez. 2025

\leavevmode\vadjust pre{\hypertarget{ref-Smith2026}{}}%
SMITH, Dave. \textbf{{OpenAI} Says It Plans to Report Stunning Annual Losses through 2028---and Then Turn Wildly Profitable Just Two Years Later}. \textbf{Fortune}, 2026. Disponível em: \textless{}\url{https://fortune.com/2025/11/12/openai-cash-burn-rate-annual-losses-2028-profitable-2030-financial-documents/}\textgreater. Acesso em: 26 jan. 2026

\leavevmode\vadjust pre{\hypertarget{ref-Vaswani2017a}{}}%
VASWANI, Ashish \emph{et al.} \href{https://proceedings.neurips.cc/paper/2017/hash/3f5ee243547dee91fbd053c1c4a845aa-Abstract.html}{Attention Is All You Need}. \textbf{Advances in neural information processing systems}, v. 30, 2017.

\leavevmode\vadjust pre{\hypertarget{ref-Wang2025}{}}%
WANG, Echo \emph{et al.} \href{https://www.reuters.com/business/openai-lays-groundwork-juggernaut-ipo-up-1-trillion-valuation-2025-10-29/}{Exclusive: {OpenAI} Lays Groundwork for Juggernaut {IPO} at up to \$1 Trillion Valuation}. \textbf{Reuters}, out. 2025.

\end{CSLReferences}

\end{document}
