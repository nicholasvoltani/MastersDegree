\documentclass[12pt]{article}
\usepackage{graphicx} % Required for inserting images
\usepackage[portuguese]{babel}
\usepackage{url}
\usepackage{hyperref}

\usepackage[alf]{abntex2cite}
\bibliographystyle{abntex2-alf}


\title{Renascendo das cinzas [\textbf{ou: ao vencedor, as batatas}]: análise marxista das bolhas das ponto-com e da inteligência artificial}
\author{Nicholas Funari Voltani}
\date{21 de dezembro de 2025}

\begin{document}

\maketitle

\section{Comparação Cisco vs NVidia}
Ambos são players centrais nas respectivas bolhas, pivotais às respectivas \textit{gold rushes}. \cite{Kollar2025}

Houve capital fixo mobilizado quando da bolha das ponto-com, o qual permaneceu após a crise --- a chamada \textit{dark fiber} --- que foi avidamente empregada em ímpetos posteriores do capital. 

Analogamente, haverão muitos \textit{data centers} inoperantes pós-crash da atual bolha de IA, o que fará o valor deste capital cair vertiginosamente. Em certo momento, o capital terá um preço atrativo a novas empreitadas --- e, como ele já está mobilizado e ``pronto para uso'', seria um ``\textit{desperdício}'' não empregá-lo.

\section{Conclusões}
O capitalista perde a cupidez com que enxergava seus meios de produção, ou melhor, sua \textit{propriedade} --- seja ela física ou de \textit{papers}. Não consegue senão olhar com desilusão para o passado que o trouxe ao presente desfiladeiro, e com desespero para a queda contra a qual cambaleia --- alguns mais, outros menos.




\bibliography{texto_bolhas.bib}

\end{document}