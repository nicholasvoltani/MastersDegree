\documentclass[12pt]{article}
\usepackage{graphicx} % Required for inserting images
\usepackage[portuguese]{babel}
\usepackage{url}
\usepackage[hidelinks]{hyperref}
\usepackage{csquotes}
\usepackage[backend=biber, style=abnt]{biblatex}
\usepackage{bookmark}
\addbibresource{Artigo_ESG.bib}



%\usepackage[alf]{abntex2cite}
%\bibliographystyle{abntex2-alf}

\title{O chão é lava: o ESG e a ética capitalista}
\author{Nicholas Funari Voltani}
\date{30 de janeiro de 2026}

\begin{document}

\maketitle

\section{}
\%\% ESG é uma versão meramente formal de sustentabilidade

A legalidade de horas extras não necessariamente é uma
reivindicação proletária \%\% \# Introdução // O que é ESG? A
crise climática que aflige o mundo no século XXI naturalmente traz o
desejo de, ou o desespero por, formas mais sustentáveis de convívio do
homem com a natureza, dentre as quais o combate ao desmatamento,
mitigação de emissões de gases de efeito estufa, a busca por uma
economia \emph{net-zero} de carbono, etc.

Mas, ao mesmo tempo, a sociedade capitalista pressupõe,
sub-repticiamente, uma ontologia conservadora sobre a conformação do
mundo em que habita, qual seja, de que o mundo nada mais é do que aquilo
que pode ser observado --- e tudo aquilo que não pode ser observado,
simplesmente não pode \emph{ser} ---, o que Bhaskar chama de ``realismo
empírico''.\footnote{Cf. (BHASKAR, 1978).} Ou seja, ignora tanto
fenômenos Dimensão Efetiva da Realidade\textbar efetivos mas
não-observáveis quanto Dimensão Real da Realidade\textbar os
mecanismos geradores dos fenômenos efetivos, escanteando-os como
``metafísicos'', ``ideais''.

As consequências lógicas de uma ontologia empirista decorrem de seu
enclausuramento em um mundo não mais que empírico --- portanto, de um
mundo \emph{fechado} --- são que a ciência torna-se algo meramente
instrumental, e que os fenômenos sociais não têm como ser algo a mais do
que as ações individuais.

É neste contexto todo que elucida-se melhor onde o ESG insere-se
juntamente às contradições do mundo capitalista, e como seus anseios
(porventura genuínos) de mudança sustentável são afogados por sua alma
indelevelmente capitalista.

\section{A apologética capitalista e sua ontologia
implícita}\label{a-apologuxe9tica-capitalista-e-sua-ontologia-impluxedcita}

Cf. (MEDEIROS, 2013a)\ldots{}

\begin{itemize}
\item
  Realismo empírico
\item
  Ciência instrumental
\item
  Imutabilidade e naturalidade do status quo
\item
  Atomismo social
\end{itemize}

\section{Ciência instrumental}\label{ciuxeancia-instrumental}

Geração de \emph{targets} sem nenhum \emph{télos}. Metas como 1.5ºC,
alcançar \emph{net-zero} até o ano 20XY, são metas tão esdrúxulas e
irreais, que \emph{nenhum shareholder} acharia admissível que os
superiores dos \emph{targets} de seus investimentos estabelecesse-os
como KPIs. Porém, como no meio corporativo, não há espaço para
questionamentos --- ou melhor, não há \emph{tempo} para questionamentos
---, e, assim como toda meta estúpida que é dada aos \emph{managers},
faz-se o que é possível e joga-se a batata quente para o próximo
desafortunado abaixo na hierarquia corporativa.

O trabalhador corporativo --- de alto ou baixo escalão --- não tem tempo
nem energia mental para debater-se com preocupações
``filosóficas''/``metafísicas'' do que concerne seu trabalho, devido à
condensação de sua jornada de trabalho a constante estado de alerta,
desgastante ao corpo, mente e alma.\footnote{``Além do esforço dos
  órgãos que trabalham, a atividade laboral exige a vontade orientada a
  um fim, que se manifesta como atenção do trabalhador durante a
  realização de sua tarefa, e isso tanto mais quanto menos esse
  trabalho, pelo seu próprio conteúdo e pelo modo de sua execução, atrai
  o trabalhador, portanto, quanto menos este último usufrui dele como
  jogo de suas próprias forças físicas e mentais.'' (MARX, 2017, p.~256)}
Ora, e nem precisa ter tal tempo, posto que a experiência atesta-lhe que
debater-se com tais questões é um beco sem saída: as vontades do mercado
serão priorizadas, e tanto o trabalhador moralmente consciente quanto o
contentemente ignorante serão forçados a aderir à vontade do mercado,
abdicando (nem que parcialmente) de seus ideais --- e, se não o fizerem,
o mercado sempre consegue contratar novos subordinados com menos
escrúpulos.

\section{O moralismo individualista e sua falsa
complacência}\label{o-moralismo-individualista-e-sua-falsa-complacuxeancia}

\begin{quote}
``Em vez de agir de maneira coletiva e contestadora, os indivíduos
comprometidos com uma transição ecológica justa podem se `engajar' como
investidores e direcionar seus recursos para um fundo ESG. Já é a
financeirização não apenas da economia, do Estado e da política social,
mas também do ativismo.'' (TELÉSFORO, 2025, p.~83)
\end{quote}

Um suíço ecologicamente consciente em ser \emph{net-zero} em sua
importantíssima vida cotidiana, indo ao trabalho de bicicleta e colhendo
os próprios temperos em casa, evidentemente não desfaz os danos de todo
o lixo tecnológico despejado em Agbogbloshie, Ghana. A vida minimalista
de um austero japonês comprometido em gerar o mínimo de lixo possível
não desfaz a gigantesca ilha de lixo de mais de 1.6 milhões de
quilômetros quadrados (!) que perambula pelo Oceano Pacífico.

(Por tratar-se de uma Ontologia implícita \emph{empirista},
o que conta é a \emph{ação}, não a intenção, dos agentes; é dada,
portanto, rédea solta ao \emph{greenwashing}, já de bate-pronto!)

Inclusive, o próprio \emph{greenwashing} é uma Estratégia
Dominada\textbar Estratégia Dominante. Suponha-se que haja alguma
estratégia de \emph{business} que esteja enquadrada como
``\emph{non-compliant} aos princípios ESG'', mas que, infelizmente, seja
saudável à saúde do portfólio; então, a empresa que finge que não segue
esta estratégia --- e, em verdade, a segue --- tem mais a ganhar (ou
melhor, menos a perder) do que os demais que realmente restringem seus
raios de ação e, portanto, têm seus negócios prejudicados.\footnote{``Seus
  interesses exclusivamente privados [do indivíduo autointeressado,
  `baconiano'] o colocam frequentemente em oposição aos interesses dos
  demais indivíduos e da sociedade como um todo: ele precisa explorar o
  trabalho e competir por fatias de mercado; lhe é interessante evadir
  impostos quando puder, apropriar-se dos recursos naturais finitos e
  consumi-los agora, sem pensar no depois; em sua operação econômica
  usual, contribui para a falência alheia enquanto cuida da sua
  prosperidade. Ao mesmo tempo, ele depende de uma estrutura
  institucional estável que preserve e proteja a propriedade privada,
  garanta a circulação global de mercadorias, prepare a população para o
  trabalho e mantenha os meios de produção inacessíveis a ela,
  tornando-a sujeita ao consumo de mercadorias. Ele também depende da
  atividade econômica de outros sujeitos privados como ele, tanto para
  satisfazer suas próprias necessidades imediatas quanto para conseguir
  as matérias-primas e outros recursos fundamentais à sua produção
  (\ldots) Em resumo: \textbf{a prática de cada um dos empreendedores
  privados se dá como se ele tivesse que esperar o comportamento mais
  equânime de todos os demais, exceto dele mesmo. Se tal comportamento
  pode ser realmente esperado, o melhor que se pode fazer continua sendo
  assumir a postura menos equânime possível, em comparação à dos
  demais.}'' (DE OLIVEIRA, p.~75; grifo meu)} Ou seja, no final do dia,
embora o homem busque ter sua consciência limpa, o dinheiro em si não
tem cheiro nem manchas\footnote{``Como a mercadoria desaparece ao se
  transformar em dinheiro, neste não se percebe como ele chegou às mãos
  de seu possuidor ou qual mercadoria foi nele transformada. O dinheiro
  \emph{non olet} [não fede], seja qual for sua origem. Se[,]
  por um lado[,] ele representa mercadoria vendida, por outro
  representa mercadorias compráveis.'' (MARX, 2017, p.~184). Embora,
  nesta citação, Marx esteja tratando, a rigor, da compra e venda de
  \emph{mercadorias}, o mesmo facilmente aplica-se ao mercado
  financeiro. É o conveniente dando as mãos ao agradável.}. \# Mas não
existe ESG bom? Sobre prevalência de ativos armamentícios em fundos ESG:
Assim como na guerra e no amor, tudo vale no \emph{business}.

Como bons capitalistas com espírito empreendedor, ``comem seus
sapos''\footnote{Expressão famosa nas comunidades de
  empreendedorismo/\emph{self-development}: ``\emph{eat your frogs
  first}''.} antes de sentarem-se à mesa para fazer negócios ``de
verdade'', bem sabendo que seus ativos verdes mais trarão retornos
reputacionais do que financeiros. Tiradas do caminho as
responsabilidades burocráticas, partem para a alocação ótima de seus
portfólios, onde vale tudo, desde tabaco até \emph{health care}, desde
armas até \emph{death care}.

Não é à toa que fundos verdes são tão recheados de ativos armamentícios
==colocar citações de Telésforo==, ou que mineradoras tão facilmente
jactem-se como sustentáveis\footnote{``O Rio? É doce // A Vale? Amarga.
  // Ai, antes fosse // mais leve a carga.'' (Lira Itabirana, Carlos
  Drummond de Andrade)}.

\section{A (insustentável) adaptabilidade do
homem}\label{a-insustentuxe1vel-adaptabilidade-do-homem}

Já tive o desprazer de ver uma ``ilustríssima'' palestrante dissertar,
em suas palestras semanais sobre ESG, em que ela, mui conhecedora de
filosofia da ciência, disse --- em meio ao ceticismo que rondava a COP
27 sobre a assertividade da meta de +1.5ºC de temperatura global
pós-revolução industrial --- que ``\emph{acreditava} que alcançaríamos
1.5ºC''. Não bastasse sua esperança e evidente rechaço da Falácia
Epistêmica\footnote{``This consists in the view that statements
  about \emph{being} can be reduced to or analysed in terms of
  statements about \emph{knowledge}; i.e.~that ontological questions can
  always be transposed into epistemological terms. The idea that being
  can always be analysed in terms of our knowledge of being, that it is
  sufficient for philosophy to `treat only of the network, and not what
  the network describes', results in \emph{the systematic dissolution of
  the idea of a world} [\ldots] \emph{independent of[,] but
  investigated by[,] science}.'' (BHASKAR, 1978, p.~36-7; grifo
  meu). Ou, em termos mais claros: ``o que pensa que eu sou, se não sou
  o que pensou?''.}, também professou seus conhecimentos geológicos,
dizendo que ``a Terra já aguentou 5 extinções em massa'' e ``sempre saiu
mais fértil delas'', então, se estamos em uma extinção em massa agora, e
a Terra já vai fazer o dela e sair mais fértil disso\ldots{} então
façamos o nosso, oras! \emph{Onde há riscos, há oportunidade}. A
necessidade é a mãe da invenção, e por sorte também é a mãe dos lucros.

A consideração que o capital concede à crise climática (para dizer
sobre uma letra do ESG apenas) não tem como ser senão \emph{passiva}:
tudo que lhes foge ao alcance financeiro torna-se mera pré-condição de
seus \emph{valuations}, e tudo que foge do horizonte de maturidade de
seus ativos não lhes diz respeito em absoluto --- \emph{après moi, le
déluge}! O capital não se importa com estimativas soltas de como o mundo
estará em 2100, e sim em como ele estará daqui a alguns poucos anos ---
e, ironicamente, a janela temporal de sua paciência em reaver seus
investimentos tem se estreitado ao longo das décadas justamente devido à
crise climática.

E não só quanto à faceta ambiental. A questão social (e/ou
socioambiental) também é vista como se estivesse detrás de uma vitrine
de loja: distante ao toque, mas não da carteira. É uma pena que o mundo
gire em função do trabalho escravo no Sudeste Asiático, mas, já que já
estão ali, por que não desfrutarmos? A possibilidade de mudança de
condições de opressão que, ``pelo bem ou pelo mal'', também são úteis e
rentáveis ao resto do mundo é recebida não só com cansaço ---
``\emph{there is no alternative}'', ``\emph{so it goes}'' etc ---, como
também, e principalmente, de forma reativa, na defensiva --- ``então
você é contra o progresso/desenvolvimento?'', ``então devemos voltar a
ser pobres?'', etc\footnote{Sempre cai como uma luva alguma menção do,
  entre aspas, ``comunismo'' como um espantalho de como o mundo
  ``poderia, mas não deveria, ser'': pobre, escasso, estanque,
  ditatorial, etc.}.

Aqui também entra o Racismo Ambiental, até mesmo em causas
``nobres'' como a erradicação da dengue através do desenvolvimento de
mosquitos geneticamente modificados:  
\begin{quote}
  ``Hence, although it
promotes itself as a very innovative strategy, the RIDL [Release of
Insects carrying a Dominant Lethal gene] transgenic insect technique
follows a deep-rooted logic that focuses on the mosquito, rather than
analyzing and improving social conditions, health care or medical
interventions. This is obvious in the Brazilian case. The town of
Juazeiro in the Northeast state of Bahia -- the chosen location to
experiment this cutting edge technology -- does not have running water
supply.'' (REIS-CASTRO, HENDRICKX, 2013 p.~124)
\end{quote}

Mosquitos esses que são propriedade\ldots{} europeia: ``\emph{All the
transgenic insects released in the environment are a product of the
British company Oxitec -- the Oxford Insect Technologies, a spin-off
company from Oxford University. [\ldots] Up to now [circa 2013],
mosquitoes have been released in the Cayman Islands, in Malaysia, and in
Brazil.}'', (Ibid., p.~119) --- destaque-se que fala-se destes insetos
como \emph{produtos} (como os autores o reafirmam ao longo do texto)! E,
no fim das contas, mesmo esses \emph{produtos} tinham uma vida mais
confortável do que seus servos humanos:

\begin{quote}
``To maintain standardized production[,] the genetically engineered
mosquitoes were given the same fish-food that was used in the English
laboratory in which they were developed; but in Brazil it was an
expensive imported brand. Moreover, the fish-food needed to be ground
twice and later sifted to turn it into a very fine powder. These steps
ensured that there were no clumps, so it could be precisely measured and
more easily dissolved in the water. While they [the workers
involved] went through this arduous and messy process, Jonatan
remarked, ``All this imported food and we need to go through all this
effort to feed them.'' After a short reflective pause, he said, shaking
his head, ``These mosquitoes have a better life than I have!''''
(REIS-CASTRO, 2021, p.~331).
\end{quote}

Dentre os maiores pecados capitais do capitalista está o desperdício de
capital, independente de que seu funcionamento cause disrupções
socioambientais. Por sua própria definição, capital consiste em meios
pelos quais a produção futura de riquezas será maior que sua produção
vigente no presente; dessa forma, a destruição de capital não só deixa
de efetivar essa produção futura, quanto destrói ``riqueza'' presente
--- é, portanto, ``um desperdício''! Afinal, para que consertar o que
``não está quebrado'', i.e.~o que \emph{ainda funciona}? Não é difícil
entender por que os apelos às ``iniciativas próprias'' que os princípios
ESG pedem ao capital foram, no máximo, inócuas: se o pecado
original\footnote{E.g. Acumulação Primitiva do Capital.} já
foi cometido, então de que vale comportar-se bem?, de que vale tentar
voltar ao Éden? Se já temos tantas plataformas de petróleo montadas,
funcionando e trazendo lucros e dividendos e \emph{royalties}, para que
diabos vamos desmontá-las? Os óbvios interesses financeiros podem até
mesmo mascarar-se de moralidade: para que vamos diminuir o ``bem-estar''
do resto do mundo?

Não é à toa, portanto, que o discurso de \emph{adaptabilidade} --- e
sinônimos próximos, como \emph{robustez}, \emph{resiliência} --- tem
tanta prevalência no vocabulário capitalista, embora possa apresentar-se
de duas formas sutilmente distintas: adaptabilidade do capital já
montado para novas funções (ou seja, de um capital ainda em
funcionamento), e adaptabilidade do mercado em incorporar novas
atividades e possibilidades (logo, instalação de \emph{mais} capital).

\section{Conclusões}\label{conclusuxf5es}

\begin{quote}
``Mais do que mero instrumento de propaganda empresarial, o discurso e o
aparato institucional ESG foram construídos explicitamente como
alternativas a iniciativas regulatórias que impõem custos ao capital ou
reduzem o seu poder. Enquanto algumas frações do capitalismo global
professam o negacionismo climático puro e simples, outras adotam
estratégias diferentes: \emph{reconhecem o problema retoricamente, mas
garantem que seu enfrentamento se restrinja às medidas voluntárias do
mercado}. O ESG é uma ferramenta de disputa de hegemonia da política
climática pelo capital financeiro global, servindo à propagação da
racionalidade neoliberal como apta e legítima para dirigir a economia, o
Estado e a sociedade.'' (TELÉSFORO, 2025, p.~82)
\end{quote}

Evidentemente há uma questão de Falácia da Composição aqui:
o comprometimento individual não garante que haja um comprometimento
coletivo. Assim como dizer que ``a humanidade'' é a principal causadora
da atual crise climática não implica que todos os seres humanos tenham a
mesma parcela de culpa, a ação meramente individual não muda o fato de
que a crise climática é causada social e sistematicamente --- oposto a
individual e contingentemente ---, não meramente pela sociedade, mas
pela sociedade \emph{capitalista} em particular.

Uma das maiores tragédias do Processo de Produção
Capitalista\textbar Modo de Produção Capitalista é que a
concorrência impele o Capital Constante\textbar Capital
individual à ação mais conveniente para si e detrimental para os demais,
ao mesmo tempo que ``socializa as perdas'' e as responsabilidades pelas
ditas ``Externalidade\textbar externalidades''; por mais que
a crise climática seja causada pelo \emph{capitalismo} como totalidade,
ela certamente é mais atribuível aos grandes conglomerados financeiros e
--- como mais recentemente tornou-se a nova \emph{auri sacra fames} ---
a sede por água e a fome por eletricidade dos \emph{data centers}, do
que do estadunidense médio (que não é o mais austero consumidor, diga-se
de passagem!).

Para o investidor, o mundo não consiste em mais do que dados e
dashboards que informam sua alocação de capital; para o capital, são só
negócios, alguns mais lucrativos, outros menos, alguns mais voláteis,
outros menos. De certa forma, o pensamento liberal está certo quando diz
que o livre-mercado trouxe a liberdade, a igualdade e a fraternidade.
Afinal, quantos honrados investidores que famintamente abstêm-se hoje de
financiar o genocídio em Gaza não suspirarão de alívio quando seu
dinheiro puder, \emph{finalmente}, içar suas âncoras ante a
``valorosíssima'' reconstrução que Israel ``humildemente'' edificará
sobre os escombros da Palestina! Afinal de contas, poucas coisas no meio
corporativo dão o mesmo gozo do que clamar seu merecido galardão sob o
clamor de trombetas\ldots{} mesmo que uma delas eventualmente abra um
abismo sob seus pés.\footnote{``E o quinto anjo tocou a sua trombeta, e
  vi uma estrela que do céu caiu na terra; e foi-lhe dada a chave do
  poço do abismo. E abriu o poço do abismo, e subiu fumaça do poço, como
  a fumaça de uma grande fornalha, e com a fumaça do poço escureceu-se o
  sol e o ar.'' (Apocalipse 9:1-2)}





  
\section{ESG como artifício retórico e hegemônico --- Artigo sobre
ESG}\label{esg-como-artifuxedcio-retuxf3rico-e-hegemuxf4nico-artigo-sobre-esg}

Dessa forma, ESG é mais uma manifestação dessa `'pasteurização'' que o
capitalismo perpetra. (O capitalismo mercadorifica ímpetos
antissistêmicos; ESG é uma versão meramente formal de
sustentabilidade). Como a ``cheia do mainstream'' (POSSAS, 1997)
parece fazer na economia, como os fenômenos de
\emph{récuperation}\footnote{Cf.
  \href{https://en.wikipedia.org/wiki/Recuperation_(politics)}{Wikipedia}:
  ``the process by which
  \href{https://en.wikipedia.org/wiki/Political_radicalism}{politically
  radical} ideas and images are twisted,
  \href{https://en.wikipedia.org/wiki/Co-opt}{co-opted}, absorbed,
  defused, incorporated, annexed or commodified within
  \href{https://en.wikipedia.org/wiki/Media_culture}{media culture} and
  bourgeois society, and thus become interpreted through a neutralized,
  innocuous or more socially conventional perspective.''.}.

No fim das contas, parece estar a par da noção de Realismo
Capitalista (cabe ler logo!).

\subsection{Relação com o caráter revolucionário da Ciência \& realismo
empírico}\label{relauxe7uxe3o-com-o-caruxe1ter-revolucionuxe1rio-da-ciuxeancia-realismo-empuxedrico}

Todo agir humano, por ser finalístico por essência\footnote{Pôr
  Teleológico.}, pressupõe algum conhecimento do funcionamento dos
objetos de seu trabalho.

\begin{quote}
``Se, de um lado, o agir humano não pode dispensar o conhecimento
correto das causas materiais sobre e com as quais ele opera e, de outro,
tem por objetivo provar esse conhecimento, então ela [a ciência]
possui a capacidade de facultar uma prática que efetivamente amplie o
grau de liberdade humana'', através da melhor possibilidade de domínio
de alguma faceta do mundo. Essa capacidade da ciência de conferir aos
seres humanos um domínio mais amplo sobre o mundo, natural e social, é
em si o seu conteúdo emancipatório.'' (MEDEIROS, p.~101)
\end{quote}

O avanço do Capital estabelece, contraditória porém
imanentemente, uma hipotrofia do caráter revolucionário da Ciência.
Contraditoriamente, pois o capital requer que a técnica e a tecnologia
sejam o mais e eficientes possível\footnote{Como Duayer o coloca: ``O
  pôr a finalidade pressupõe, afirma Lukács, uma apropriação espiritual
  da realidade orientada pelo fim posto, pois só dessa maneira o
  resultado do trabalho pode ser algo novo, algo que não emergiria de
  maneira espontânea dos processos próprios da natureza. No entanto, por
  contraste, assinala Lukács, o reordenamento dos materiais e processos
  naturais requerido para que eles possam dar origem ao fim posto
  \emph{exige um conhecimento o mais adequado possível desses objetos e
  processos}, precisamente por convertê-los de legalidades (processos)
  naturais em legalidades postas. Ao contrário do antropomorfismo
  próprio da possessão espiritual da realidade condicionada pela
  finalidade planejada [sic], aqui há de prevalecer o máximo de
  desantropomorfização, pois a consecução do fim não seria possível sem
  o conhecimento das propriedades dos objetos e processos envolvidos na
  transferência das causalidades naturais em causalidades postas.'\,'
  (DUAYER, p.~127; grifo meu).} para obter sucesso em valorizar-se;
imanentemente, pois a própria Ontologia que a sociedade
capitalista pressupõe implicitamente é um Realismo Empírico,
em que o mundo é totalmente saturado pelos fenômenos Dimensão
Empírica da Realidade\textbar Empíricos, havendo, portanto, uma
naturalização e concreteza nas estruturas que já existem, em que a
ciência assume um caráter instrumental, empunhada e brandida pelo
capital e seus interesses.

\subsection{Realismo empírico -> Atomismo social}

\begin{quote}
``O atomismo social, para expressá-lo sinteticamente, nada mais é do que
a concepção do mundo do realismo empírico. Em outras palavras,
considerar que a realidade social seja exaurida pelos fenômenos
empíricos é equivalente a admitir que esta realidade é constituída
exclusivamente por indivíduos e suas ações, \emph{já que estas são as
únicas coisas efetivamente empíricas existentes na sociedade}.''
(MEDEIROS, p.~122-3; grifo no original)
\end{quote}

\subsection{Sobre empregos \& dispersão da
responsabilidade}\label{sobre-empregos-dispersuxe3o-da-responsabilidade}

A grande tragédia do capitalismo é que tantos empregos aparentem ser
inócuos --- e que sejam, de fato, imediatamente inócuos ---, mas que
estejam ``distantemente'' ao uso de ações mais visivelmente nefastas.
Trabalhos como análises de padrões em dados numéricos, classificação dos
dados de clientes de alguma empresa, etc., são ações aparentemente
triviais: o analista de dados pode facilmente não fazer ideia do que
seus dados representam e fazer bem seu trabalho; os pilotos dos aviões
dos quais foram jogadas as bombas de Hiroshima e Nagasaki não foram
contratados para serem assassinos, e sim pilotos de avião, não mais.
Essa desconexão, que sequer precisa ser dissonância cognitiva do
trabalhador, ocorre justamente porque seu trabalho é meramente formal no
que tange à sua essência: a extração de Mais-Valor para seu
contratante.

\subsection{Sobre o escândalo na Suíça de seus investimentos no Sul
Global}\label{sobre-o-escuxe2ndalo-na-suuxeduxe7a-de-seus-investimentos-no-sul-global}

Ocorre, portanto, uma tragédia em dois níveis. Há o escândalo
superficial, embora espalhafatoso, de europeus ante atividades ilegais
em que seus investimentos \emph{porventura} esbarraram, um escândalo
porquanto agora todos enxergam que o \emph{repugnante} rei está nu, o
que evidentemente não orna bem com o chá da tarde. Ultrajados que o
mundo não é limpo, puro, justo, como ele (obviamente) é na Europa. Que o
mundo está se recusando à ``normalidade'' de não dar notícias ruins, e
sim de bons resultados de negócios, bons rendimentos e \emph{business as
usual}. Quem sabe se sentem-se mais ultrajados de \emph{saber} do lado
negro de seus negócios do que das consequências em si.

E há a resignada realização de que o mercado, por si só, \emph{talvez}
não consiga resolver todos os problemas a que se propõe. Os burros, por
assim dizer, já foram à água, e não se pode impedir que eles a bebam.

O caso do ESG é um caso paradigmático da contradição entre valores e
ontologia. Deseja-se que haja maior sustentabilidade, combate à crise
climática, menos desmatamento etc. etc., mas, ao mesmo tempo,
pressupõe-se a ontologia conservadora do capitalismo, seu
Empirismo e naturalização das estruturas vigentes,
pressupostos ou conscientemente --- ``\emph{there is no alternative}''
--- ou acriticamente, em cujo talvez o façam até mais ferreamente, posto
que sequer sabem (ou querem saber) de suas restrições.

O ESG ilustra bem a precedência ontológica que o estado das coisas ---
estruturas naturais e sociais --- possuem ante os julgamentos que delas
são feitos, de tal forma que os ``ideais por uma nova economia'' são
cooptados pela própria realidade que (supostamente) pretendiam reformar,
reduzindo-se a meras reproduções do status quo (embora travestidas de
seu oposto).


\printbibliography

\end{document}