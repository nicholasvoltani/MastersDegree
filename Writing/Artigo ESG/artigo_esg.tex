\documentclass[12pt]{article}
\usepackage{graphicx} % Required for inserting images
\usepackage[portuguese]{babel}
\usepackage{url}
\usepackage[hidelinks]{hyperref}
\usepackage{xurl} % must be loaded AFTER hyperref
\usepackage{csquotes}
\usepackage[backend=biber, style=abnt]{biblatex}
\usepackage{bookmark}
\usepackage{epigraph}
\addbibresource{Artigo_ESG.bib}
\setlength{\epigraphwidth}{0.5\textwidth}


%\usepackage[alf]{abntex2cite}
%\bibliographystyle{abntex2-alf}

\title{A esquizofrenia do ESG: entre a ética do capital e a ética da vida}
\author{Nicholas Funari Voltani}
\date{30 de janeiro de 2026}

\begin{document}
\maketitle

\epigraph{``...\textit{for certainly, you were better take for business a man somewhat absurd than over-formal.}'' (Francis Bacon)}




\section{Introdução}
A crise climática que aflige o mundo no século XXI naturalmente traz o desejo de, ou o desespero por, formas mais sustentáveis de convívio do homem com a natureza, dentre as quais o combate ao desmatamento, mitigação de emissões de gases de efeito estufa, a busca por uma economia \textit{net-zero} de carbono\footnote{Ou seja, em que todas as emissões de carbono são contrabalançadas por pela capacidade do planeta Terra de absorvê-las.}, etc.

Mas, ao mesmo tempo, a sociedade capitalista pressupõe, sub-repticiamente, uma ontologia conservadora sobre a conformação do mundo em que habita, qual seja, de que o mundo nada mais é do que aquilo que pode ser observado --- e tudo aquilo que não pode ser observado, simplesmente não pode \textit{ser} ---, o que Bhaskar chama de “realismo empírico” \cite{Bhaskar1978}. Ou seja, a ontologia pressuposta pela sociedade capitalista ignora tanto fenômenos efetivos mas não-observáveis quanto os mecanismos geradores dos fenômenos efetivos, escanteando-os como “metafísicos”, meramente “ideais”.

As consequências lógicas de uma ontologia empirista decorrem de seu enclausuramento em um mundo não-mais-que-empírico --- portanto, de um mundo ``\textit{fechado}'' --- são que a ciência torna-se algo meramente instrumental, e que os fenômenos sociais não têm como ser algo a mais do que as ações individuais.

É neste contexto todo que elucida-se melhor onde o ESG insere-se juntamente às contradições do mundo capitalista, e como seus anseios (porventura genuínos) de mudança sustentável são afogados por sua alma indelevelmente capitalista.







\section{A doutrina Friedman}
Em seu famoso artigo de 1970 pelo New York Times, Milton Friedman faz seu argumento contra as ``responsabilidades sociais'' que começavam a ser imputadas ao homem de negócios, estabelecendo, dessa forma, a assim chamada `` doutrina Friedman''. Encapsulada na última frase de sua publicação, ele afirma que, em uma ``sociedade livre'',
\begin{quote}
    ``há uma e somente uma responsabilidade social dos negócios --- usar os recursos destes e engajar-se em atividades destinadas a aumentar seus lucros, desde que estejam \textit{dentro das regras do jogo}, ou seja, engajar-se em competição aberta e livre, sem enganação ou fraude.'' \cite[grifo meu]{Friedman1970}
\end{quote}

O eixo central de sua argumentação é que um executivo corporativo, tendo sido apontado pelos acionistas (\textit{shareholders}) de uma empresa, torna-se responsável pela gestão e administração dos ativos destes, e, naturalmente, de acordo com \textit{os interesses destes}. Dessa forma, caso deseje engajar-se em ``responsabilidades sociais'' de forma que elas contraponham-se aos interesses de seus acionistas --- quais sejam, essencialmente a geração de lucros ---, então ele está ``gastando o dinheiro deles'' --- ou, mais explicitamente, está desperdiçando-o, está agindo de má fé para com aqueles que apontaram-no para esta responsabilidade.








\section{O trabalho corporativo}

\begin{quote}
    ``...mesmo o próprio trabalho comercial que, à primeira vista, parece desprovido de atração [além do ``ganho'' financeiro], dá muitas vezes \textit{um grande prazer oferecendo um objetivo ao exercício das faculdades do homem e a seus instintos de emulação e de poder}[,] pois, assim como um cavalo de corrida ou um atleta, que exige tudo de cada um dos seus nervos para exceder seus concorrentes, e sente prazer nesse esforço, assim também um industrial ou um comerciante é muitas vezes estimulado mais pela esperança de vencer seus rivais do que pelo desejo de juntar algo à sua fortuna.'' \cite[p. 39; grifo meu]{Marshall1985}
\end{quote}

Os altos salários já dão boa conta das necessidades materiais em geral, dando margem ao trabalhador corporativo para buscar alguma satisfação que seja neste trabalho que, em sua essência, não lhe diz respeito\footnote{``Além do esforço dos órgãos que trabalham, a atividade laboral exige a vontade orientada a um fim, que se manifesta como atenção do trabalhador durante a realização de sua tarefa, e isso tanto mais quanto menos esse trabalho, pelo seu próprio conteúdo e pelo modo de sua execução, atrai o trabalhador, portanto, quanto menos este último usufrui dele como jogo de suas próprias forças físicas e mentais.'' \cite[p. 256]{Marx2017}}. Conforme adequa-se à realidade ao \textit{habitus} do meio em que trabalha, começa a perceber, mesmo que tacitamente, suas regras e seu ``espírito''. É neste sentido que mesmo o trabalhador assalariado corporativo, uma ínfima peça deste sistema ``que ele encontra pronto'', é capaz de personificar a concorrência capitalista. O que os move, satisfeitas suas necessidades e preocupações com sua sobrevivência, tem a liberdade de ser o \textit{thrill of the hunt}, a adequação de agir conforme as ``regras do jogo'' do mercado, de vencer pela astúcia, seja ela \textit{fair play} ou um \textit{foul play} aceitável institucionalmente (ou em suas despesas).

Os salários dos neófitos deste meio têm, também, de ser altos o suficiente, para que possam ser sua porta de entrada neste mundo que lhes é alheio; os salários de seus gerentes têm de ser altos o suficiente para que não sejam a razão de eles procurarem a porta de saída da empresa, para que não sintam-se alheios na empresa onde estão. Para os primeiros, é necessidade; para os últimos, é suficiência. Os primeiros agem em nome da empresa, mas em prol de si mesmos; os últimos agem por si mesmos, mas em prol da empresa\footnote{Embora venha a ``vestir a camisa'' de sua empresa, o executivo corporativo, como Frank Underwood, ainda ``reza \textit{para} si mesmo, \textit{por} si mesmo'' [``\textit{I pray} to \textit{myself}, for \textit{myself}''] (\textit{House of Cards}, T1E12). Embora ele até possa esquecer-se disso, iludindo-se com a própria máscara que pôs no rosto, ele é eventualmente forçado a aceitar essa egoísta realidade quando a empresa desfaz-se dele, análogo a como ele desfez-se de sua própria vida pessoal em prol de ``sua'' empresa. O capital é prudente o suficiente de não apegar-se demasiado a nenhuma das ferramentas que empregue, sejam elas de carne ou aço.}.

Dessa forma, o trabalhador corporativo, de alto ou baixo escalão, é capaz de incorporar a ética do mercado tão aderentemente que até tornam-se capazes de ``entrar em fluxo'' com ele, de sintonizar-se com ele, de \textit{entender} ele e as ações que são \textit{adequadas} a se tomar. Alguns chegam até mesmo a crer que suas ações, por serem tão paralelas ao que espera-se do mercado, moldam \textit{diretamente} o mercado; chegam a entreter a fantasia de que é o indivíduo que molda a sociedade de punho próprio, e que esta não consiste em mais do que a soma de seus espécimes e de suas ações --- embora saibam, bem no fundo, irrefletidamente, que falar ``do mercado'' é algo além de falar ``de seus demais concorrentes'' ou da soma dos \textit{players} de um setor da indústria. Isso se manifesta quando falam, por exemplo, de \textit{business cycles}, que \textit{impõem-se às empresas} e \textit{forçam sua adaptação}, não o contrário!\footnote{``O comportamento do homem esgota-se, portanto, no cálculo correto das oportunidades desse curso [\textit{Ablauf}: curso, processo] (cujas ``leis'' ele já encontra ``prontas''), na habilidade de evitar os ``acasos'' perturbadores [\textit{störender »Zufälligkeiten«}] por meio da aplicação de dispositivos de proteção e medidas defensivas (que se baseiam igualmente na consciência e na aplicação de ``leis'' semelhantes); muitas vezes, chega até mesmo a se deter no cálculo das probabilidades dos possíveis efeitos de tais ``leis'', sem sequer tentar intervir no próprio processo pela aplicação de outras ``leis'' (como nos esquemas de segurança etc.). Quanto mais se considera essa situação em profundidade e independentemente das lendas burguesas sobre o caráter ``criador'' dos expoentes da época capitalista, tanto mais claramente aparece, em tal comportamento, a analogia estrutural com o comportamento do operário em relação à máquina que ele serve e observa, e cujo funcionamento ele controla enquanto a contempla.'' \cite[p. 218-9; destaque meu]{Lukacs2003a}.}







\section{A ética do capital}
A ética do capital tem por \textit{télos} sua autovalorização, priorizando, portanto, os modos de ação que conduzam-no à sua \textit{raison d'être}. É neste sentido que essas falas acima mencionadas, de tais marechais do exército do capital, expõem a hipocrisia do ESG no meio corporativo: não como uma denúncia de sua negligência para com a sociedade, e sim de sua negligência para com os negócios; não porque suas ações --- que restringem-se a precificações --- são só ``para inglês ver'', e sim porque são uma perda de tempo, recursos e pessoal \textit{no que tange ao próprio capital}. 

Portanto, não criticam o ESG porque sua alma é antissocial, e sim porque ela é anti-capital. Como conciliar as demandas por mudança social se a disrupção da sociedade vem da opressão do capital? Pela sua hegemonia, o capital só cumprirá sua ``responsabilidade social'' por ``boa vontade'', o que é o mesmo que dizer ``por motivos ulteriores'', pois sabe que age contra seu espírito ao fazê-lo, sabe que peca perante seu Deus-dinheiro e que precisa seguir Seu evangelho até o fim dos tempos.

A própria concorrência impele os capitais individuais a não se preocuparem com o futuro tão-tão-distante de alguns anos adiante\footnote{``Nosso prazo médio de empréstimo [na HSBC] é 6 anos. O que acontece com o planeta no 7º ano é irrelevante para nossa carteira de crédito.'' (Stuart Kirk, \textit{op. cit.}, 11:05). ``Quem hoje está investindo e analisando prefere não se preocupar porque não sabe se estará daqui a algumas décadas na mesma área ou sequer vivo, a pessoa dá o check favorável para a empresa e pronto.'', diz Alperowitch em \cite{Angelo2022}.}, pois são os capitais de hoje com que disputa-se no Valhalla que é o mercado. [em que lutam as mais aguerridas batalhas de dia, e sentam-se à mesa juntos, para o jantar, após o fechamento da Bolsa.]

\textbf{Adicionar anotações de Lukács} 








\section{Ambiental, Social, Governança: o paradigma ESG/ASG}
Sobre Stuart Kirk:\footnote{É com este causo que Telésforo introduz seu texto em \cite{Telesforo2025}.}
\begin{itemize}
    \item Discurso no evento do Financial Times, FT Moral Money Summit \cite{Kirk2022} (20 de maio de 2022)
    \item O próprio CEO do group HSBC, Noel Quinn, posta no seu Linkedin --- num sábado! --- sobre seu rechaço ao discurso de Kirk no dia anterior \url{https://www.linkedin.com/posts/noel-quinn-hsbc_i-do-not-agree-at-all-with-the-remarks-activity-6933729465193668608-14ho?utm_source=social_share_send&utm_medium=member_desktop_web&rcm=ACoAADGDblYBTpg3cdnCuk-ijUGghBmcJcxAlLc}
    \item Kirk é suspenso da HSBC em 22 de maio de 2022 \cite{Makortoff2022}
    \item Kirk se demite da HSBC em 7 de julho de 2022 \cite{Reuters2022}
    \item Kirk é contratado em novembro de 2022... pelo próprio \textit{Financial Times}! \cite{FinancialTimes2022}
\end{itemize}


\url{https://www.linkedin.com/posts/noel-quinn-hsbc_i-do-not-agree-at-all-with-the-remarks-activity-6933729465193668608-14ho?utm_source=social_share_send&utm_medium=member_desktop_web&rcm=ACoAADGDblYBTpg3cdnCuk-ijUGghBmcJcxAlLc}


Ora, se Kirk realmente fosse considerado um desvairado por seus pares, ele não teria sido contratado pelo próprio jornal --- respeitado na área de Economia e finanças --- em cujo evento ele ``disse barbaridades''! 

E --- que surpresa! --- o HSBC já recuou em seus compromissos ESG, frente à pressão, abre aspas, ``anti-\textit{woke}'' que os EUA assumiram após a posse de Trump em 2025: protelou seu louvável objetivo de alcançar o mágico \textit{net-zero} de 2030 para 2025 \cite{Segal2025a}, e saiu do grupo \textit{Net-Zero Banking Alliance} \cite{Segal2025}. Tudo isso no contexto do segundo mandato de Trump e sua cruzada contra qualquer coisa relacionada com ``inclusão'', ``sustentabilidade'' etc. Isso já é relatado em \cite{Mirza2025}, dez dias após ação presidencial da Casa Branca para, abre aspas, ``Acabar com programas/preferências radicais e onerosos de DEI\footnote{Diversidade, Igualdade e Inclusão (\textit{Diversity, Equity and Inclusion}).} [\textit{Ending Radical And Wasteful Government DEI Programs And Preferencing}]'' \cite{WhiteHouse2025}.

Uma coisa é cortar a visibilidade destes programas para o público, mas continuá-los internamente. Outra é ver isso como uma oportunidade de expandir seus negócios --- como o corporativo adora fazer. 



%Tomemos um exemplo: suponhamos que o CEO de algum banco tenha sido influenciado pelo temível ``esquerdismo'' e descubra as atrocidades cometidas em nome do agronegócio, decidindo que não quer mais financiar suas atrozes atividades. Suponhamos então que nosso honrável executivo formaliza algum plano concreto que corte todos seus negócios com o agronegócio (e suponhamos que ``agronegócio'' seja bem delimitado dentro do portfolio do banco). Naturalmente, além do ultraje que seus clientes do agronegócio sentiriam, também alguns de seus \textit{outros} clientes poderiam tornar-se temerosos: e se \textit{minha} área de atividade for a próxima a ser cortada? Resultado: provável fuga indireta de clientes, \textit{bank runs}. E além disso, e talvez mais importante, a \textit{imensa} depreciação do valor de mercado deste banco. 

Pode até ser que ``tudo valha no amor e na guerra``, mas isso certamente não vale para os negócios. Não é preciso ler o artigo de Friedman para saber que este CEO teria seus dias contados no banco (quem dirá em sua carreira), mas, como Friedman colocou-o tão candidamente em sua publicação, vale ler dele próprio:
\begin{quote}
    ``E, mesmo que ele [o executivo corporativo] queira, ele consegue mesmo se safar de gastar o dinheiro de seus acionistas, clientes ou empregados? Não iriam esses acionistas demiti-lo? (Seja os [acionistas] atuais ou aqueles que assumirão depois que suas [ó, tão nobres!] ações em nome da responsabilidade social tenham reduzido os lucros da corporação e o preço de suas ações.) Seus clientes e empregados o deserdariam por outros produtores e empregadores menos escrupulosos [!] em exercer suas responsabilidades sociais [\textit{their social responsabilities}].'' \cite[destaques meus]{Friedman1970}
\end{quote}

Isso até mesmo os homens de negócios do século XVII já sabiam, como bem resumido por Francis Bacon: ``É certamente melhor convidar para os negócios um homem algo obtuso [\textit{somewhat absurd}] do que um que seja formal demais [\textit{over-formal}]''\footnote{Tradução de \cite[p. 64; destaques meus]{deOliveira2023}.}.

Restaurada a ordem e estabelecido algum executivo mais adequado à consecução devida dos negócios, espera-se que ele “coma seus sapos”\footnote{Expressão famosa nas comunidades de empreendedorismo/self-development: “eat your frogs first”: tirar as coisas difíceis logo do caminho.} antes de sentar-se à mesa para fazer negócios “de verdade”, bem sabendo que seus ativos verdes mais trarão retornos \textit{reputacionais} do que financeiros. Tiradas do caminho as responsabilidades sociais/burocráticas, partem para os negócios propriamente ditos: a alocação ótima de seus portfólios, onde vale tudo, desde tabaco até \textit{health care}, desde armas até \textit{death care}.Não é à toa, afinal, que fundos verdes são tão recheados de ativos armamentícios, ou que mineradoras tão facilmente jactem-se como sustentáveis.\footnote{\textbf{COLOCAR CITAÇÕES DE TELÉSFORO.}. “O Rio? É doce // A Vale? Amarga. // Ai, antes fosse // mais leve a carga.” (Lira Itabirana, Carlos Drummond de Andrade)}

\begin{quote}
    ``O investimento sustentável é uma área confusa das finanças, que muitas vezes significa coisas diferentes para pessoas diferentes. Na maioria das vezes, significa construir carteiras de investimento que excluem categorias questionáveis, como o "desinvestimento" em produtores de combustíveis fósseis, numa aparente tentativa de combater as mudanças climáticas. Infelizmente, há uma diferença entre se eximir de algo em que você não deseja participar e lutar ativamente contra algo que você acredita que precisa parar para o bem de todos.'' \cite[p. 9]{Fancy2021}
\end{quote}

E, no fim das contas, \textit{much ado about nothing}\footnote{``Tanto estardalhaço por nada'' (Shakespeare).}: manchetes sobre ``o investidor anti-clima'' bombaram, obviamente colocando-o como o errado da história,enquanto aqueles que estão no meio financeiro \textit{sabem} que ele está certo, \textit{sabem} da contradição em termos que é ``finanças sustentáveis'', \textit{sabem} que ele diz o que lhes é evidente todos os dias em suas rotinas de trabalho.
\begin{quote}
    ``...a realidade é que grande parte do que importa para a sociedade simplesmente não afeta o retorno de uma estratégia de investimento particular. Muitas vezes, isso se deve ao horizonte temporal do investimento subjacente: muitas estratégias têm um prazo muito curto, o que significa que as questões ESG de longo prazo não são particularmente relevantes. Às vezes, existem estratégias de dívida que possuem vários seguros e proteções [\textit{insurance and protections}] e que, de qualquer forma, não se preocupam com um período posterior à data de vencimento da dívida. Na maioria das vezes, investimentos caros e de longo prazo relacionados à sustentabilidade são incertos e demoram para gerar potencial de lucro, se é que existe algum.'' \cite[p. 15]{Fancy2021}
\end{quote}

\begin{quote}
    ``As pessoas que tomam essas decisões de alocação de capital são treinadas para reagir rapidamente às mudanças nos rendimentos esperados e na rentabilidade das oportunidades de investimento. Se você dissesse a um gestor de portfólio para reduzir a pegada de carbono de seu portfólio, a maioria concordaria com a cabeça e diria que é importante, mas depois voltaria a alocar capital para os empreendimentos mais rentáveis --- exatamente como são legalmente obrigados e financeiramente incentivados a fazer. Se, por outro lado, uma externalidade negativa for ``internalizada'' pelo governo por meio de, digamos, um imposto sobre a poluição, reduzindo assim a rentabilidade dos grandes emissores de gases de efeito estufa, os alocadores de capital \textit{automaticamente} reagirão o mais rápido possível, alocando menos capital para essas oportunidades agora menos rentáveis.

    ``A maioria dos gestores de portfólio sabe intuitivamente que tudo isso acontece. Um gerente de projetos da BlackRock foi explícito comigo: “Eu acredito nas mudanças climáticas. Se tivéssemos um preço para o carbono, eu reduziria minha pegada de carbono da noite para o dia --- e todos os outros também. Mas não faz sentido fazer isso sozinho e me colocar em desvantagem, e não é o que eu sou legalmente obrigado a fazer ou pelo que sou pago.”'' \cite[p. 24]{Fancy2021}
\end{quote}


\begin{quote}
    ``Muitos dos funcionários de nível operacional [\textit{rank and file employees}] em empresas que geram lucro de maneiras que prejudicam o público não se dão conta disso. Frequentemente, isso ocorre porque trabalham em silos [i.e. isoladamente]. Ou porque o problema é complexo. E de longo prazo. Às vezes, tudo isso junto. Para a alta administração [\textit{senior management}], pode ser tentador acreditar em uma fantasia que se alinha perfeitamente aos seus interesses financeiros. Como disse, certa vez, o escritor americano Upton Sinclair: "É difícil fazer um homem entender algo quando seu salário depende de ele não entendê-lo".'' \cite[p. 26]{Fancy2021}
\end{quote}








\section{A insustentável adaptabilidade do homem}
A consideração que o capital concede à crise climática não tem como ser senão passiva: tudo que lhe foge ao alcance financeiro torna-se mera pré-condição de seus \textit{valuations}, e tudo que foge do horizonte de maturidade de seus ativos não lhes diz respeito em absoluto --- \textit{après moi, le déluge}\footnote{``Depois de mim, o dilúvio'', expressão atribuída ao rei francês Luís XV.}! O capital não se importa com estimativas soltas de como o mundo estará em 2100, e sim em como ele estará daqui a alguns poucos anos. (Ironicamente, a janela temporal de sua paciência em reaver seus investimentos tem se estreitado ao longo das décadas justamente devido à crise climática.)

E não só quanto à faceta ambiental. A questão social (e/ou socioambiental) também é vista como se estivesse detrás de uma vitrine de loja: distante ao toque, mas não da carteira. É uma pena que o mundo gire em função do trabalho escravo no Sudeste Asiático, mas, já que já estão sendo empregados, \textit{por que não desfrutarmos de seus produtos}? A possibilidade de mudança de condições de opressão que, ``pelo bem ou pelo mal'', também são úteis e rentáveis ao resto do mundo, é recebida não só com cansaço — ``\textit{there is no alternative}'', ``\textit{so it goes}'' etc —, como também, e principalmente, de forma reativa, na defensiva — ``então você é contra o progresso/desenvolvimento?'', ``então devemos voltar a ser pobres?'', etc.\footnote{Sempre cai como uma luva, nessas reações defensivas, alguma menção do, entre aspas, ``comunismo'' como um espantalho de como o mundo ``poderia, mas não deveria, ser'': escasso, pobre, estático, ditatorial, etc. Como se já não vivêssemos num mundo assim!}

A questão do racismo ambiental sempre está à espreita das discussões de ESG, e, mesmo quando é explicitamente reconhecida, dificilmente é tomada a sério. É claro: porque essa questão é tacitamente pressuposta por tantos \textit{business as usual}. Um exemplo que o atesta claramente pode ser encontrado no trabalho antropológico de Luisa Reis-Castro, que fez um trabalho de campo em Juazeiro (Bahia) relacionado à liberação de mosquitos geneticamente modificados, uma estratégia de erradicação do \textit{Aedes aegypti}. Essa causa ``nobre'', que facilmente pode ser vista como ``cumprindo princípios ESG'', é mais complexa do que parece:
\begin{quote}
    ``Assim, embora se promova como uma estratégia muito inovadora, a técnica de insetos transgênicos RIDL [sigla para ``Liberação de Insetos Portadores de um gene letal Dominante''] segue uma lógica profundamente enraizada que se concentra no mosquito, em vez de analisar e melhorar as condições sociais, saúde e atendimentos médicos. Isso fica evidente no caso brasileiro. A cidade de Juazeiro, no estado da Bahia --- local escolhido para experimentar essa tecnologia de ponta --- \textit{não possui abastecimento de água encanada} [!].'' \cite[p. 124; grifo meu]{Reis-Castro2013}
\end{quote}

Mosquitos esses que são criados por uma empresa, com o propósito de serem comercializados... ou seja, são mercadorias! Mercadorias essas produzidas por uma empresa... britânica: Oxitec, um braço comercial da Universidade de Oxford.

E, não bastasse isso, esses mosquitos tinham uma vida melhor que os trabalhadores locais que os ``produziam'':
\begin{quote}
    ``Para manter a produção padronizada, os mosquitos geneticamente modificados receberam a mesma ração para peixes usada no laboratório inglês onde foram [originalmente] desenvolvidos; porém, no Brasil, tratava-se de uma marca importada e cara. Além disso, essa ração precisava ser moída duas vezes e peneirada para se transformar em um pó bem fino. Essas etapas garantiam que não se formassem torrões, permitindo uma dosagem precisa e uma dissolução mais fácil na água. Enquanto os trabalhadores envolvidos realizavam esse processo árduo e trabalhoso, Jonatan [pseudônimo de um dos trabalhadores locais nesse laboratório] comentou: “Toda essa ração importada e todo esse esforço para alimentá-los”. Após uma breve pausa reflexiva, balançando a cabeça, ele disse: “Esses mosquitos têm uma vida melhor que a minha!”.''\cite[p. 331]{Reis-Castro2021}
\end{quote}

Dentre os maiores pecados capitais na catequese do capitalista está o desperdício de capital, independente de seu funcionamento causar disrupções socioambientais. Por sua própria definição, capital consiste em meios pelos quais a produção futura de riquezas será maior que sua produção vigente no presente; dessa forma, a destruição de capital não só deixa de efetivar essa produção futura, quanto destrói ``riqueza'' presente — é, portanto, ``um desperdício''! Mais que isso, pode-se concluir: porquanto é uma destruição de riqueza já existente, é um ``crime''! Afinal, para que consertar o que ``não está quebrado'', i.e. o que \textit{ainda funciona}? Não é difícil entender por que os apelos às ``iniciativas próprias'' que os princípios ESG pedem ao capital foram, no máximo, inócuas: se o dano já está feito, então de que vale comportar-se bem? Se já temos tantas plataformas de petróleo montadas, funcionando e trazendo lucros e dividendos e \textit{royalties}, para que diabos vamos desmontá-las? Nesses casos, os óbvios interesses financeiros podem até mesmo mascarar-se de moralidade: para que vamos diminuir o ``bem-estar'' do resto do mundo?\footnote{``Por exemplo, uma CEO pode decidir reduzir a pegada de carbono de sua empresa, mas não pode fazê-lo simplesmente por ser ``justo'' ou ``a coisa certa a se fazer''; a decisão precisa ser justificada em termos dos interesses dos acionistas.'' \cite[p. 16]{Fancy2021}}

Não é à toa, portanto, que o discurso de \textit{adaptabilidade} --- e sinônimos próximos, como robustez e resiliência --- tem tanta prevalência no vocabulário capitalista: onde há riscos, há oportunidade; a necessidade é a mãe da invenção, e, por sorte, também é a mãe dos lucros. Porém, o discurso de ``adaptabilidade'' apresenta-se de duas formas sutilmente distintas: adaptabilidade do capital já montado para novas funções, ou seja, de um capital ainda em funcionamento (logo, um uso mais arraigado deste capital); e adaptabilidade do mercado em incorporar novas atividades e possibilidades (logo, instalação de capital \textit{novo}).







\section{Considerações finais}
\begin{quote}
    ``Mais do que mero instrumento de propaganda empresarial, o discurso e o aparato institucional ESG foram construídos explicitamente como alternativas a iniciativas regulatórias que impõem custos ao capital ou reduzem o seu poder. Enquanto algumas frações do capitalismo global professam o negacionismo climático puro e simples, outras adotam estratégias diferentes: \textit{reconhecem o problema retoricamente, mas garantem que seu enfrentamento se restrinja às medidas voluntárias do mercado}. O ESG é uma ferramenta de disputa de hegemonia da política climática pelo capital financeiro global, servindo à propagação da racionalidade neoliberal como apta e legítima para dirigir a economia, o Estado e a sociedade.''\cite[p. 82; grifo no original]{Telesforo2025}
\end{quote}

Evidentemente há uma questão de Falácia da Composição aqui: o comprometimento individual não garante que haja um comprometimento coletivo. Assim como dizer que ``a humanidade'' é a principal causadora da atual crise climática não implica que todos os seres humanos tenham a mesma parcela de culpa, a ação meramente individual não muda o fato de que a crise climática é causada social e sistematicamente --- oposto a individual e contingentemente ---, não meramente pela sociedade, mas pela sociedade \textit{capitalista} em particular.

Uma das maiores tragédias do modo de produção capitalista é que a concorrência impele o capital individual à ação mais conveniente para si e detrimental para os demais, ao mesmo tempo que “socializa as perdas” e as responsabilidades pelas ditas “externalidades”; por mais que a crise climática seja causada pelo capitalismo como totalidade, ela certamente é mais atribuível aos grandes conglomerados financeiros e — como mais recentemente tornou-se a nova \textit{auri sacra fames} — a sede por água e a fome por eletricidade dos data centers, do que do estadunidense médio (que não é o mais austero consumidor, diga-se de passagem!).

Para o investidor, o mundo não consiste em mais do que dados e \textit{dashboards} que informam sua alocação de capital; para o capital, são só negócios, alguns mais lucrativos, outros menos; alguns mais voláteis, outros menos. Mas, no fim das contas, todo mundo prefere trabalhar acreditando que está fazendo um bem maior. Afinal, quantos honrados investidores que famintamente abstiveram-se até agora de financiar o genocídio em Gaza suspirarão de alívio quando seu dinheiro puder, \textit{finalmente}, içar suas âncoras e fluir diretamente à “valorosíssima” reconstrução que os Estados Unidos “humildemente” edificarão sobre os escombros da Palestina!\footnote{Trump, em 04/02/2025, disse, em entrevista coletiva, que haveria uma ``oportunidade fenomenal e \textit{unbelievable}'' na Palestina, a ``\textit{Riviera do Oriente Médio}'' --- na qual os palestinos \textit{também} iriam viver \cite{CNBCTelevision2025}. No Fórum Econômico de Davos de 2026, onde Trump anunciou o \textit{Board of Peace} com seus Estados lacaios -- quer dizer, aliados --, Jared Kushner (intermediário nas relações Israel-Gaza e genro de Donald Trump) anunciou o ``plano mestre'' para a reconstrução de Gaza, recortando seu território como se fosse uma planta de urbanismo, com áreas de turismo litorâneo, complexo industrial, parques e áreas residenciais \cite{Haddad2026}. Convenientemente, Kushner é um homem de negócios do setor... imobiliário.} 

Afinal de contas, poucas coisas no meio corporativo dão o mesmo gozo do que clamar seu merecido galardão sob o clamor de trombetas… mesmo que uma delas eventualmente abra um abismo sob seus pés.\footnote{“E o quinto anjo tocou a sua trombeta, e vi uma estrela que do céu caiu na terra; e foi-lhe dada a chave do poço do abismo. E abriu o poço do abismo, e subiu fumaça do poço, como a fumaça de uma grande fornalha, e com a fumaça do poço escureceu-se o sol e o ar.” (Apocalipse 9:1-2)}


%% \section{A finitude}
%% O capital tem ojeriza à finitude, à restrição. Vê obstáculos não como limites que o impedem, e sim como fronteiras que o instigam sua travessia. Preocupa-se, porém, com o que está logo em sua frente, um problema de cada vez; restringe suas preocupações mediante desconto hiperbólico, e segue sua precificação do \textit{business as usual}. 

%% O que acontece no mundo, o que acomete seus clientes humanos, só diz respeito ao capital porquanto afetar seus negócios: se é positivo a seus clientes, faz mais; se é negativo, se ajusta; e se é oneroso, repensa.


\printbibliography

\end{document}