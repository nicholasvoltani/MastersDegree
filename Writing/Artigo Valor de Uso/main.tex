\documentclass[12pt]{article}
\usepackage{graphicx} % Required for inserting images
\usepackage[portuguese]{babel}
\usepackage{url}
\usepackage{hyperref}
\usepackage{mathtools}
\hypersetup{
    colorlinks= true,
    citecolor = black,
    linkcolor = black,
    urlcolor = black
    }
\usepackage{csquotes}


\usepackage[alf]{abntex2cite}
\bibliographystyle{abntex2-alf}

\title{Artigo Valor de Uso (\textbf{Esboço})}
\author{Nicholas Funari Voltani}
\date{27 de novembro de 2025}

\begin{document}

\maketitle

\begin{abstract}
    Um conhecido resultado d'\textit{O Capital} de Marx é que as mercadorias ``não possuem um só átomo'' de valor, ou seja, que o valor de uma mercadoria pertence à dimensão social, não sendo algo imediatamente empírico; a partir disso, leituras vulgares costumam atribuir todo o caráter material de uma mercadoria à sua faceta de valor de uso. Neste trabalho, discute-se como a característica de valor de uso de um objeto, apesar de ser ontologicamente resultado de sua ``corporiedade'' material, trata-se (também) de uma categoria social.
\end{abstract}


% \section{}
% Para Marx, os seres humanos executam \textit{trabalho} quando buscam satisfazer suas necessidades materiais --- ``do estômago ou da imaginação'' ---, agindo sobre a natureza de forma intencional. A questão da \textit{intencionalidade da ação} é importante, pois é isso, como o próprio Marx o coloca, o que diferencia ``o pior arquiteto da melhor abelha'':
% \begin{quote}
%     ``No final do processo de trabalho, chega-se a um resultado que já estava presente na representação [mental] do trabalhador no início do processo, ou seja, um resultado que já existia \textit{idealmente}.'' \cite[p. 256; grifo meu]{Marx2013}
% \end{quote}

% Essa é, portanto, uma característica ``transistórica'' do ser humano, ``eterna necessidade natural de mediação do metabolismo entre homem e natureza''
% \footnote{\cite[p. 120]{Marx2013}.}. 
% Para os propósitos deste trabalho, não faz-se necessário adentrar tanto nessa (riquíssima) discussão da categoria trabalho em Marx.\footnote{Uma boa recomendação quanto à questão ontológica da categoria marxiana do trabalho pode ser encontrada em \cite{Medeiros2016}, onde explora-se a rica convergência das ideias de Roy Bhaskar com as de György Lukács quanto a esta temática.}

% Dessa forma, o resultado do trabalho, um objeto que foi trazido ao mundo para a satisfação de necessidades humanas, é dito ser um \textit{valor de uso}. 

% Ademais, Marx escreve que ``descobrir os valores de uso das coisas é um ato histórico''\footnote{``Toda coisa útil (...) deve ser considerada sob um duplo ponto de vista: o da qualidade e o da quantidade. Cada uma dessas coisas é um conjunto de muitas propriedades e pode, por isso, ser útil sob diversos aspectos. Descobrir esses diversos aspectos e, portanto, as múltiplas formas de uso das coisas é um ato histórico.'' \cite[p. 113]{Marx2013}}: dizer que certos objetos possuem valor de uso \textit{não} quer dizer que sua \textit{raison d'être} natural, antediluviana, é a de ser valor de uso. Maçãs não são do jeito que são pura e simplesmente para que seres humanos as comam, e sim o inverso: seres humanos as usufruem (comem, consomem) justamente porque suas propriedades (p. ex. acidez, doçura, textura) são de seu interesse, satisfazem certas necessidades suas (fome, desejo). De fato, valores de uso, por sua própria definição, são sempre valores de uso \textit{no que concerne a seres humanos}.\footnote{Talvez seja pertinente destacar que o usufruto de valores de uso não se restringe ao seu consumo \textit{imediato}, mas também de seu uso conjuntamente com outros valores de uso: o óleo de cozinha pode ser um ingrediente importante para a satisfação da fome, mas não é bom para matar a sede.}

% Bhaskar argumenta que \textit{fenômenos sociais são inseparáveis de seus efeitos sociais} \cite[pp. 44-54]{Bhaskar1998}, e isso está claro para o valor de uso e o valor de uma mercadoria: ela só é valor [de uso] conquanto há uma expectativa ``concreta'' de que ela \textit{efetive} sua potência --- caso contrário, ela não será algo útil, e portanto não será um valor de uso, e valor tampouco. 

% O valor de uso de um objeto pressupõe que ele possa ser \textit{meio de satisfação de necessidades humanas} (caráter universal), e que efetive tal capacidade \textit{em algum caso particular}. \textit{Mutatis mutandis} para seu valor: espera-se que ele possa \textit{efetivamente} servir, para seu detentor, como um valor [relativo] a ser trocado por algum [valor] equivalente de outrem, e que os objetos recebidos após a troca sejam úteis para seus novos portadores. 

% Cotidianamente, quando se fala de uma maçã, não se pensa nos elementos químicos que a constituem, em sua espécie e sua ``evolução'' ao longo do tempo; pensa-se no seu formato e cor, sim, mas, por baixo de sua materialidade, sempre pensamos nela, ulteriormente, como \textit{comida}. 

% Ou seja, parece haver uma ``redundância notacional'': confunde-se a maçã-enquanto-objeto-social com a maçã-enquanto-objeto-natural. Dessa forma, fica mais evidente que seu ser \textit{natural} possui precedência ontológica quanto a seu ser \textit{social}. Ou seja, com o fim da humanidade haveria um \textit{armageddon} de valores de uso\footnote{Pois valores de uso são úteis \textit{para seres humanos}, por pressuposto, e, com a inexistência de seres humanos, não é possível existirem valores de uso (argumento por vacuidade).}, mas não (necessariamente) de seus substratos naturais. 

% É o velho \textit{ceci n'est pas une pipe}, porém sutil. O valor-de-uso-social é, no fundo, o valor-de-uso-natural --- pois ``reside'' nesta materialidade, necessariamente ---, mas é algo a mais: está aquém, em poder causal, de sua ``fisicalidade'' própria. Um valor de uso, \textit{enquanto objeto natural}, pode ser algo ``estanque'', mas, enquanto objeto social, possui certa \textit{dinâmica} ao longo do tempo -- leia-se: ao longo da história social. Por exemplo, uma mesma maçã pode ter sido mero meio de alimentação há milênios atrás, mas, hoje em dia, é também ingrediente em várias culinárias ao redor do mundo; ou seja, o mesmíssimo objeto (ainda que com gostos e espécies diferentes ao redor do mundo) adquiriu poderes causais distintos, a ele imbuídos pela sociedade ao longo do tempo e do espaço, muito embora ``nenhum átomo'' de valor de uso tenha sido adicionado a seu corpo material.

% Ou seja, a sociedade imbui este objeto com um novo caráter (social), e este objeto --- agora, ao menos em parte, social --- induz com que a sociedade utilize-o (para satisfação de necessidades) e imbua-lhe novos caráteres com o tempo, em particular quando coadunado com \textit{outros} valores de uso, a fim de satisfazer outras necessidades, em geral, mais complexas (como satisfação de fome vs. dar um certo sabor específico a pratos).

% Abstraindo, portanto, dos caracteres materiais de valores de uso, resta somente sua ``utilidade''\footnote{Um caso desafortunado da cooptação  semântica dos economistas, pois não faz sentido chamar de ``função utilidade'' algo que quantifica um \textit{estado de satisfação} --- utilidade é algo \textit{transitivo}, algo só pode ser útil \textit{para algum propósito}!}: matar a fome, saciar a sede, resguardar do frio, fornecer abrigo, aprimorar conhecimentos, permitir comunicação à distância, etc. É possível falar destas necessidades sem fazer menção ao caráter qualitativo específico através do qual foram/podem ser satisfeitas: p. ex. a fome pode ser satisfeita com carne de primeira ou carne de segunda.

% Ademais, a categoria valor de uso é algo eminentemente \textit{histórico}, não só porque trata-se de características descobertas em certos períodos da história humana, mas porque abarca usos, potências, que perseveram no tempo, ou melhor, cujas potências são conhecidas e preservadas ao longo de gerações humanas. Paralelamente, a complexidade que as necessidades humanas adquirem com o tempo tornam a categoria valor de uso algo mais complexo: ela é histórica e, por consequência, devém algo social para que seja \textit{sabidamente} útil. Um exemplo prático: há uma espécie de peixe no Japão que tem de ser cozinhado minuciosamente para que suas toxinas deixem de ser tóxicas ao (corajoso, ou temerário) consumidor final. A descoberta deste peculiar valor de uso gastronômico certamente é algo histórico --- descoberto em certo período da história humana --- e necessariamente \textit{social}, posto que requer a manipulação deliberada de elementos naturais a fim de que se obtenha uma ``utilidade'' desejada/desejável. 

% Como bem colocado por Duayer,

% \quote{``O pôr a finalidade pressupõe, afirma Lukács, uma apropriação espiritual da realidade orientada pelo fim posto, pois só dessa maneira o resultado do trabalho pode ser algo novo, algo que não emergiria de maneira espontânea dos processos próprios da natureza. No entanto, por contraste, assinala Lukács, o reordenamento dos materiais e processos naturais requerido para que eles possam dar origem ao fim posto exige um conhecimento o mais adequado possível desses objetos e processos, precisamente por convertê-los de legalidades (processos) naturais em legalidades postas. Ao contrário do antropomorfismo próprio da possessão espiritual da realidade condicionada pela finalidade planejada [sic], aqui há de prevalecer o máximo de desantropomorfização, pois a consecução do fim não seria possível sem o conhecimento das propriedades dos objetos e processos envolvidos na transferência das causalidades naturais em causalidades postas.'' \cite[p. 127]{Duayer2023a}}

% Ou seja, é através do ato humano que as ``legalidades naturais'' tornam-se uma ``legalidade'' humanamente útil; em outras palavras, as características naturais dos objetos são condições (ontologicamente) necessárias para que sejam valores de uso, \textit{mas não são condições suficientes}!

% \section{}
% \begin{quote}
%     ``A utilidade de uma coisa faz dela um valor de uso. Mas essa utilidade não flutua no ar. Condicionada pelas propriedades do corpo da mercadoria,\footnote{No original, ``\textit{\textbf{Durch} die Eigenschaften des Waarenkörpers \textbf{bedingt}}...'' [grifo meu], i.e. ``\textbf{Dependente das/Condicional às} propriedades do corpo-mercadoria...''.} ela não existe sem esse corpo. Por isso, o próprio corpo-mercadoria [\textit{Warenkörper}], como ferro, trigo, diamante etc., é um valor de uso ou um bem. Esse seu caráter não depende do fato de a apropriação de suas qualidades úteis custar muito ou pouco trabalho aos homens.'' \cite[p. 114]{Marx2013}
% \end{quote}


% O valor de uso é, antes de tudo, um objeto \textit{concreto}\footnote{Ou, como Marx o coloca no começo d'\textit{O Capital}, ``externo'' [\textit{äußerer}].}, com propriedades intrínsecas materiais. Não tem-se que o mundo existe \textit{para que} seja usufruído pelo ser humano, e sim que as propriedades das coisas \textit{podem} (ou não) \textit{ser} de usufruto para a satisfação de necessidades humanas. (Acreditar na primeira asserção é retornar ao mito de Adão e Eva...)

% Note-se, porém, que parece estranho dizer que são ``as substâncias químicas'' da maçã as características que satisfazem necessidades humanas. 
% % Aqui aparece, em particular, a noção de características \textit{emergentes}: não há ``substâncias químicas'' da maçã em que se encontrem sua acidez, ou sua doçura, ou sua textura, pois tais características surgem \textit{na interface entre} a materialidade dos valores de uso com seu usufruto humano. O açúcar \textit{per se} não é doce; ele é doce \textit{ao paladar} (em particular, humano). 
% Tais propriedades são fenômenos \textit{emergentes} das características mais ``granulares'' (moleculares) da maçã, porém irredutíveis a estas.\footnote{Da mesma forma, tais propriedades satisfazem necessidades \textit{humanas}, mas não pode-se localizar quais células humanas possuem tal necessidade, nem como elas são saciadas; a própria noção de \textit{necessidade humana} é também um fenômeno emergente.} Da mesma forma que as propriedades da água não se reduzem às propriedades do hidrogênio e do oxigênio que a compõem, também o valor de uso de um objeto não se reduz às propriedades puramente materiais deste objeto. De novo: um valor de uso o é \textit{com relação a sua satisfação de necessidades humanas}; se parte do valor de uso da maçã é ``ser doce e agradável ao paladar'', é porque o é \textit{para o ser humano}.

% Quando se fala de um valor de uso \textit{enquanto tal} (um filósofo da ciência o escreveria como ``um valor de uso \textit{qua} valor de uso''), não pensa-se em suas características particulares, seus elementos físicos, químicos, etc; pensa-se \textit{além} delas, em sua \textit{pura potencialidade}. A potencialidade de um objeto de \textit{ser valor de uso} é algo que lhe é interno (no que concerne suas propriedades particulares), porém também é algo \textit{social}, posto que diz respeito à satisfação de certas necessidades \textit{do ser humano}; é uma potência, portanto, que lhe é própria, mas que também possui um fator ``externo'' que lhe é incutida.

% Dado um objeto que seja um valor de uso, pode-se pensá-lo por suas facetas ``objeto-natural'' e ``objeto-social''. Evidentemente todo valor de uso é objeto-natural antes de tornar-se objeto-social, mas não é como se este objeto-natural, \textit{em si e por si próprio}, adquirisse características novas só porque nós, seres humanos, assim o ensejamos. Um objeto é social necessariamente porque é natural, não o contrário\footnote{Mesmo ``bens imateriais'' são passíveis dessa decomposição. Por exemplo, uma aula online é, antes de tudo, vibrações no ar advindas da fala de algum professor, captadas e processadas por dispositivos eletrônicos, energia elétrica correndo por cabos de transmissão e \textit{data centers} em algum recôndito da Terra, e retransmitida por ondas eletromagnéticas para algum estudante, luzes emitidas por uma tela etc.; uma aula pode ser tudo isso, mas \textit{não se reduz} a isso.}.

% No fim das contas, é o velho ``\textit{ceci n'est pas une pipe}'', porém mais sutil: o objeto-social é, no fundo, o objeto-natural -- pois são tais características que são o fundamento de sua relevância ao usufruto humano --, mas torna-se agora algo ``mais-que-natural'', ou melhor, ``não-somente-natural''. Torna-se algo além no tocante ao seu poder causal perante seres humanos, mas além também no tocante à sua ``fisicalidade própria''. Ou seja, adquire características que não assentam-se sobre sua ``corporeidade'', e sim sobre a ``malha social'' em dado instante no tempo e local no espaço. 

% Um valor de uso enquanto objeto-natural pode ser algo ``estanque'', mas, enquanto objeto-social, ele torna-se algo ``dinâmico'', que adquire novas características\footnote{E, ocasionalmente, perde outras: no começo do século XX, a economia chilena foi à bancarrota quando seu suprimento de nitratos à Inglaterra -- usado para fertilizantes e explosivos -- tornou-se obsoleto frente à produção facilitada de nitrogênio via processo de Haber, cf. \cite{Foster2004}.} com o decorrer do tempo -- em particular, com o desenrolar da história humana da qual faça parte. 

% A partir do momento em que efetivam suas potencialidades, mercadorias deixam de ser valor de uso (valor).\footnote{Já aqui há um ponto de atenção no tocante à temática ambiental: quando mercadorias servem para a satisfação de ``uma única'' necessidade, segue-se que sua corporeidade se torna obsoleta por completo -- ou seja, vira lixo. Portanto, já está pressuposto, no próprio espírito do capitalismo, a rápida obsolescência das mercadorias (leia-se: a redução do tempo em que elas ``agem'' como valores de uso), e a consequente produção generalizada de lixo. ``Com a predominância sempre crescente da população urbana, amontoada em grandes centros pela produção capitalista, esta, por um lado, acumula a força motriz histórica da sociedade e, por outro lado, desvirtua o metabolismo entre o homem e a terra, isto é, o retorno ao solo daqueles elementos que lhe são constitutivos e foram consumidos pelo homem sob forma de alimentos e vestimentas, retorno que é a eterna condição natural da fertilidade permanente do solo. Com isso, [a produção capitalista] destrói tanto a saúde física dos trabalhadores urbanos como a vida espiritual dos trabalhadores rurais''  \cite[p. 572-3]{Marx2013}.}

\section{O valor de uso enquanto objeto material}
Um valor de uso é, antes de tudo, um objeto material ``externo'' [\textit{äußerer}]\footnote{\cite[p. 113]{Marx2017a}.}, com propriedades materiais intrínsecas. Costuma-se pensar que são justamente essas características ``físicas'' de um objeto que configuram seu valor de uso. Note-se, porém, que parece estranho dizer que ``as substâncias químicas'' da maçã são as características que satisfazem necessidades humanas, por mais que, tecnicamente, isso esteja correto. Contudo, mudando o exemplo, sabe-se que um fumante pode até começar a usar adesivos de nicotina, mas eles não serão a mesma coisa que cigarros; em verdade, o cigarro e os adesivos satisfarão necessidades sutilmente distintas. 

Em suma, as capacidades físico-químicas de um objeto de satisfazer necessidades humanas --- logo, de ser um valor de uso --- consistem em \textit{fenômenos emergentes}, advindos de propriedades mais ``granulares'' (e.g. moleculares) que este objeto possui, capacidades estas que são irredutíveis a estas propriedades.\footnote{Da mesma forma, tais propriedades satisfazem necessidades \textit{humanas}, mas não pode-se localizar quais células humanas possuem tal necessidade, nem como elas são saciadas molecularmente; a própria noção de \textit{necessidade humana} é também, a rigor, um fenômeno emergente.} Portanto, da mesma forma que as propriedades da água não se reduzem às propriedades do hidrogênio e do oxigênio que a compõem, também o ``ser valor de uso'' de um objeto não se reduz às propriedades puramente materiais deste objeto --- não restringe-se, portanto, à sua ``materialidade''.

Até mesmo os ditos ``bens imateriais'' são passíveis dessa decomposição. Por exemplo, uma aula online é, antes de tudo, vibrações no ar advindas da fala de algum professor, captadas e processadas por dispositivos eletrônicos, energia elétrica correndo por cabos de transmissão e \textit{data centers} em algum recôndito da Terra, e retransmitida por ondas eletromagnéticas para algum estudante, tornando-se luzes e sons emitidos por um computador ou celular etc.; uma aula pode ser composta por tudo isso, mas \textit{não se reduz} a isso. O funcionamento do setor de serviços requer coisas que já damos por dadas, como algum local físico\footnote{Só porque algumas lojas tenham fechado seus locais físicos não quer que seu estoque deixe de ocupar espaço.}, oxigênio/ar, eletricidade (e, portanto, cabos/fibras óticas), etc.\footnote{Pode ser uma surpresa a alguns que tudo que está na \textit{cloud}, em verdade, tem seus pés bem fincados na materialidade dos \textit{data centers} em que estão hospedados.}

%Todo setor de serviços de uma economia assenta-se, literalmente, sobre alguma superfície sólida e imerso em (no mínimo) oxigênio, sendo impossível, portanto, prescindir ao menos destes fatores materiais/naturais.\footnote{Satélites artificiais até podem orbitar a Terra, mas seus ``comandantes'' estão, via de regra, com o pé no chão.}

O que diferencia objetos que são valores de uso daqueles que não o são --- objetos ``úteis'' vs. objetos ``inúteis'' --- é sua \textit{relevância à satisfação de necessidades humanas}, ou melhor, sua viabilidade de satisfazer alguma necessidade humana através de seu usufruto. Se parte do valor de uso da maçã, por exemplo, é ``ser doce e agradável ao paladar'', é porque o é \textit{para o ser humano}. 

De fato, é verdade que o ser \textit{natural} de um objeto tem precedência ontológica quanto a seu ser \textit{social}, qual seja, sua potencialidade de satisfazer necessidades humanas. Afinal, um objeto é social necessariamente porque é natural, mas a recíproca não é sempre verdadeira, pois existem objetos naturais que não têm qualidade social, e.g. objetos que não são úteis ao homem. Portanto, é condição necessária, em particular, de todo valor de uso que ele seja um objeto natural, mas isso não é condição suficiente para que ele seja, de fato, \textit{útil}.






\section{Sobre ética e trabalho}
\cite{Medeiros2016}

\textbf{***Comentários sobre ética e alternativas de ação como sendo o que dita quando algum objeto material torna-se valor de uso.***}

\textbf{***Elaboração sobre caráter do trabalho no tornar um objeto material em um valor de uso, por conta de teleologia etc***}

% Evidentemente, todo valor de uso é objeto-natural antes de tornar-se objeto-social, mas não é como se este objeto-natural, \textit{em si e por si próprio}, adquirisse características novas só porque nós, seres humanos, assim o ensejamos idealmente. 






\section{Valor de uso como objeto físico criado socialmente, e tornando-se historicamente concreto}
Dessa forma, a categoria valor de uso toma um caráter \textit{histórico}, não só porque trata-se de características descobertas em certos períodos da história humana, mas porque abarca usos e potências que perseveram no tempo, ou melhor, cujo conhecimento destes é conhecido e compartilhado ao longo de gerações humanas. Dessa forma, a complexidade que as necessidades humanas adquirem com o tempo tornam a categoria valor de uso algo mais complexo: ela tornou-se uma categoria eminentemente social (ainda que sempre possua algum suporte material) e, por consequência, devém algo histórico, para que seja \textit{sabidamente} útil ao longo do tempo.\footnote{É nesse sentido que Marx diz que ``Fome é fome, mas a fome que se sacia com carne cozida, comida com garfo e faca, é uma fome diversa da fome que devora carne crua, com mão, unha e dente'' \cite[\textbf{checar página}]{Marx2021}: os valores de uso se complexificam junto das necessidades, e as necessidades se expandem mediante a complexificação dos valores de uso.}

Assim, a sociedade imbui este objeto com um novo caráter (social), e este objeto --- agora, ao menos em parte, social --- induz com que a sociedade utilize-o (para satisfação de necessidades) e imbua-lhe novos caráteres com o tempo, em particular quando coadunado com \textit{outros} valores de uso, a fim de satisfazer outras necessidades, em geral, mais complexas (como satisfação de fome vs. dar um certo sabor específico a pratos).\footnote{O usufruto de algum valor de uso não necessariamente tem de se dar isolado dos demais: o óleo de cozinha pode ser um ingrediente importante para a satisfação da fome, mas certamente não é bom para matar a sede.}

Um exemplo prático: há uma espécie de peixe no Japão, chamado \textit{fugu} (em português, chamado de baiacu), que tem de ser cozinhado minuciosamente para que suas toxinas deixem de ser tóxicas ao (corajoso, ou temerário) consumidor final. Este valor de uso gastronômico certamente é algo \textit{social}, posto que requer a manipulação deliberada de elementos naturais a fim de que se obtenha uma ``utilidade'' desejada/desejável, e necessariamente é algo \textit{histórico}, posto que não só foi descoberto em certo período da história humana, como porque requereu, e requer, a continuidade da \textit{expertise} de sua produção, a ser repassada intergeracionalmente.

Outro exemplo é a enorme variedade de espécies de milho no Peru, resultado de milênios de reprodução cruzada de plantas executada pelos povos e civilizações que ali viveram. Fica claro aqui que não foi o desenrolar da ``história natural'' da evolução dessas espécies vegetais que condicionou o desenvolvimento das civilizações andinas, e sim o contrário: foi o desenvolvimento socio-técnico que induziu tal variabilidade genética em escalas de tempo ``humanas'' (contraposto a escalas de tempo ``naturais'', de milhões de anos)\footnote{Vale bem o que Marx diz: ``Com exceção da indústria extrativa, cujo objeto de trabalho é dado imediatamente pela natureza, tal como a mineração, a caça, a pesca etc. [...], todos os ramos da indústria manipulam um objeto, a matéria-prima, isto é, um objeto de trabalho já filtrado pelo trabalho, ele próprio produto de um trabalho anterior, tal como a semente na agricultura. Animais e plantas, que se costumam considerar como produtos naturais, são, em sua presente forma, não só produtos do trabalho, digamos, do ano anterior, mas o resultado de uma transformação gradual, realizada sob controle humano, ao longo de muitas gerações e mediante o trabalho humano.'' \cite[p. 259]{Marx2017a}. Por outro lado, é evidente que fatores menos facilmente manipuláveis --- como no que tange ao esforço e às escalas de tempo envolvidos, como a geografia local, e a própria evolução natural da fauna e flora locais --- condicionaram mais ``unilateralmente'' o modo de vida desses povos.}. Em geral, pode-se dizer, sim, que a ``evolução natural'' dos hábitats influenciam no caráter social de povos que neles vivem, mas não é sempre o caso; de fato, conforme complexificam-se as relações humanas com a natureza, tende-se a inverter isso, como bem vemos pela atual crise climática.

Como bem colocado por Duayer,

\begin{quote}
    ``O pôr a finalidade pressupõe, afirma Lukács, uma apropriação espiritual da realidade orientada pelo fim posto, pois só dessa maneira o resultado do trabalho pode ser algo novo, algo que não emergiria de maneira espontânea dos processos próprios da natureza. No entanto, por contraste, assinala Lukács, o reordenamento dos materiais e processos naturais requerido para que eles possam dar origem ao fim posto exige um conhecimento o mais adequado possível desses objetos e processos, precisamente por convertê-los de legalidades (processos) naturais em legalidades postas. Ao contrário do antropomorfismo próprio da possessão espiritual da realidade condicionada pela finalidade planejada [sic], aqui há de prevalecer o máximo de desantropomorfização, pois a consecução do fim não seria possível sem o conhecimento das propriedades dos objetos e processos envolvidos na transferência das causalidades naturais em causalidades postas.'' \cite[p. 127]{Duayer2023a}
\end{quote}

Ou seja, é através do ato humano que as ``legalidades naturais'' tornam-se uma ``legalidade'' humanamente útil; em outras palavras, as legalidades naturais são condições (ontologicamente) necessárias para que sejam valores de uso, \textit{mas não são condições suficientes}!

Dessa forma, embora um valor de uso enquanto objeto-natural possa ser algo ``estanque'', enquanto objeto-social ele torna-se algo ``dinâmico'', que adquire novas características com o decorrer do tempo --- em particular, com o desenrolar da história humana da qual faça parte.\footnote{E, ocasionalmente, perde certas características: no começo do século XX, a economia chilena foi à bancarrota quando seu suprimento de nitratos à Inglaterra --- usado para fertilizantes e explosivos -- tornou-se obsoleto (ao menos ao mercado britânico) frente à produção facilitada de nitrogênio via processo de Haber, cf. \cite{Foster2004}.}

% A partir do momento em que efetivam suas potencialidades, mercadorias deixam de ser valor de uso (valor).\footnote{Já aqui há um ponto de atenção no tocante à temática ambiental: quando mercadorias servem para a satisfação de ``uma única'' necessidade, segue-se que sua corporeidade se torna obsoleta por completo -- ou seja, vira lixo. Portanto, já está pressuposto, no próprio espírito do capitalismo, a rápida obsolescência das mercadorias (leia-se: a redução do tempo em que elas ``agem'' como valores de uso), e a consequente produção generalizada de lixo. ``Com a predominância sempre crescente da população urbana, amontoada em grandes centros pela produção capitalista, esta, por um lado, acumula a força motriz histórica da sociedade e, por outro lado, desvirtua o metabolismo entre o homem e a terra, isto é, o retorno ao solo daqueles elementos que lhe são constitutivos e foram consumidos pelo homem sob forma de alimentos e vestimentas, retorno que é a eterna condição natural da fertilidade permanente do solo. Com isso, [a produção capitalista] destrói tanto a saúde física dos trabalhadores urbanos como a vida espiritual dos trabalhadores rurais''  \cite[p. 572-3]{Marx2013}.}



\section{Abstrações}
Quando se fala de um objeto \textit{qua} valor de uso,
% \footnote{Ou seja, um objeto \textit{enquanto} valor de uso.},
não pensa-se em suas características particulares, seus elementos físicos, químicos, etc; pensa-se \textit{além} delas, pensa-se em sua \textit{pura potencialidade} --- portanto, em seu caráter \textit{geral}, não particular. A potencialidade de um objeto de \textit{ser valor de uso} é algo que lhe é interno (no que concerne suas propriedades particulares de satisfação de necessidades humanas), porém também é algo \textit{social}, posto que diz respeito à satisfação, \textit{lato sensu}, de necessidades \textit{do ser humano}; é uma potência, portanto, que lhe é própria, mas que também possui um fator ``externo'' que lhe é incutido e ao qual passa a remeter.\footnote{Carcanholo define o ``valor de uso formal'' que certas ``mercadorias especiais'' --- dinheiro, força de trabalho e a mercadoria-capital --- assumem como ``utilidades que elas assumem em suas relações formais com a economia mercantil-capitalista'' \cite[p. 32-3]{Carcanholo1998}. Embora ele descreva que o que torna o dinheiro, em particular, uma mercadoria especial, enquanto equivalente geral, não é seu valor de uso ``próprio'', e sim o valor de uso ``que lhe foi outorgado pelo desenvolvimento da sociedade mercantil, das trocas, das formas do valor'' \cite[p. 34]{Carcanholo1998} --- portanto, seu valor de uso \textit{formal}, não específico ---, como busco elaborar ao longo deste trabalho, \textit{todos} os valores de uso têm caráter social, não só as mercadorias ``especiais''. O que ocorre é que estas têm não somente o caráter social de sua existência (como todas as mercadorias têm), como também o caráter social \textit{pós-usufruto}. [\textbf{Discutível se é pertinente e/ou correto.}]}
% \footnote{Referenciando uma passagem de Marx em \textit{Para uma crítica da Economia Política}, Carcanholo elabora que ``o valor de uso em si, enquanto produto ou bem abstraído de suas determinações históricas, não joga nenhum papel na economia política. A significação econômica do valor de uso só aparece na sua relação com as condições sociais de produção, tanto quando é influenciado por estas, como quando influi nessas condições. Portanto, o valor de uso formal, por assim chamá-lo quando não se apresenta unicamente em suas propriedades materiais, não pode ser tratado apenas como elemento subordinado, assim como o fazem os defensores da tese da irrelevância do valor de uso.'' \cite[p. 29]{Carcanholo1998}. Essa noção de ``valor de uso formal'', }

Abstraindo, portanto, dos caracteres materiais de valores de uso, resta somente sua ``utilidade''\footnote{Um caso desafortunado da cooptação  semântica dos economistas, pois não faz sentido chamar de ``função utilidade'' algo que quantifica um \textit{estado de satisfação} --- utilidade é algo \textit{transitivo}, algo só pode ser útil \textit{para algum propósito}!}: matar a fome, saciar a sede, resguardar do frio, fornecer abrigo, aprimorar conhecimentos, permitir comunicação à distância, etc. É possível falar destas necessidades sem fazer menção ao caráter qualitativo específico através do qual foram/podem ser satisfeitas: p. ex. a fome pode ser satisfeita com carne de primeira ou carne de segunda.

\textbf{Ou seja: abstraindo do caráter social, valores de uso são passíveis de estudos pelas ciências exatas; abstraindo do caráter físico, valores de uso são passíveis de estudo por sua utilidade ``abstrata''.}

\textbf{Contudo, não parece possível abstrair do caráter físico ao se falar de um objeto enquanto valor de uso: uma broca de cirurgia não pode ser feita de qualquer tipo de material, há materiais \textit{mais adequados}!}




\section{Relação dialética entre forma material e forma social de VUs}
\textbf{Há uma relação dialética entre ser material e ser social de um valor de uso? Quando é visto como material, pode prescindir de sua característica social; quando é visto como algo social, pode prescindir de sua forma material.}
Ou seja, parece haver uma ``redundância notacional'': confunde-se o objeto-enquanto-objeto-social com o objeto-enquanto-objeto-natural. Dessa forma, fica mais evidente que seu ser \textit{natural} possui precedência ontológica quanto a seu ser \textit{social}. Ou seja, com o fim da humanidade haveria um \textit{armageddon} de valores de uso\footnote{Pois valores de uso são úteis \textit{para seres humanos}, por pressuposto, e, com a inexistência de seres humanos, não é possível existirem valores de uso (argumento por vacuidade).}, mas não (necessariamente) de seus substratos naturais.

No fim e ao cabo, é o velho ``\textit{ceci n'est pas une pipe}'', porém sutil. O valor-de-uso-social é, no fundo, o valor-de-uso-natural --- pois ``reside'' nesta materialidade, necessariamente ---, mas é algo a mais: está aquém, em poder causal, de sua ``fisicalidade'' própria. Um valor de uso, \textit{enquanto objeto natural}, pode ser algo ``estanque'', ``trivial'', mas, enquanto objeto social, possui certa \textit{dinâmica} ao longo do tempo --- leia-se: ao longo da história social. Por exemplo, uma mesma maçã pode ter sido mero meio de alimentação há milênios atrás, mas, hoje em dia, é também ingrediente em várias culinárias ao redor do mundo (ainda que com gostos e espécies diferentes ao redor do mundo). Ou seja, o mesmíssimo objeto adquiriu novos poderes causais, a ele imbuídos pela sociedade ao longo do tempo e do espaço, muito embora ``nenhum átomo'' de valor de uso tenha sido adicionado a seu corpo material.


\section{Debate 28/11/25}
\subsection{Patrick}
Este trabalho há relação com dissertação?

R: 

---
Lukács: causalidade dada vs causalidade posta. É a mediação teórica que esclarece questão de valor de uso (material vs social). Bhaskar: caráter intransitivo,  ``pressuposto''. 


\subsection{Ian}
Argumento pode seguir do abstrato ao mais concreto, mas a forma como foi feito parecia que ``ia para um lado mas foi para o outro''. Uma leitura curta pode dar um julgamento injusto, pois o final do texto esclarece o prévio...

Exemplo da internet: a forma social também está em como um valor de uso é \textbf{empregado}, não só como foi descoberto. Ex: Internet não foi originalmente criada para que se espalhassem fake news.

Citação de Engels: olho da água enxerga mais longe que o olho humano, mas o olho humano enxerga mais coisas que o olho da águia.

Também Engels: os homens começaram a falar quando tinham algo a dizer.

Frase mal estruturada, sobre `estanque'/`dinâmico': o ser humano \textit{coloca} os nexos causais ``sobre um objeto'', não que o objeto ``adquire'' novas potências.

Sobre frase de ``valor de uso qua valor de uso''... melhor desconsiderar... (Comentário de que todo universal é um universal-particular.) 


\subsection{Patrick}
Não dá pra dissociar questão social de questão material num valor de uso, há uma unidade de fato.

Menção de Manuscritos de 1844: os sentidos humanos são um resultado de toda a história; similar ao comentário de Engels da águia.

Exemplo de esportes e do desenvolvimento de técnicas e tecnologias que tornam-os mais \textit{sociais}, no sentido de que seu ``desenvolvimento'' depende cada vez menos da questão \textit{física} dos atletas.

\subsection{Ian}
``As potencialidades de um objeto de satisfazer necessidades etc. emergentes'': parece que tais características emergentes são algo ``determinístico'', não algo pertinente socialmente e até advindo do socialmente.

\subsection{Patrick}
Averiguar questão material \textbf{e social} da Energia/mercadoria-energia. Referência: Sokal e Bricmann, Imposturas intelectuais. Pode auxiliar na questão da crítica acima.

Ontologia II: fala do caráter social da educação, diferenciada de ``educação'' animal. Diferencia-se em agir de modo distinto, teleologicamente, preparando para o futuro. 

Discussão de epifenômenos: macaco-prego faz canivetes com pedra!! Foi passando biologicamente (???)



\bibliography{Artigo_Valor_de_Uso}

\end{document}